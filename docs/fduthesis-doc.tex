% \begin{documentation}
%
%^^A 用户手册的页边距
%^^A+
% \newgeometry{
%   left   = 1.65 in,
%   right  = 0.80 in,
%   top    = 1.25 in,
%   bottom = 1.00 in
% }
%
%^^A-
%
% \section{介绍}
%^^A! \section{Introduction}
%^^A!
%
% 目前,在网上可以找到的复旦大学 \LaTeX{} 论文模板主要有以下这些:
% \begin{itemize}
%   \item 数学科学学院 2001 级的何力同学和李湛同学在 2005 年根据
%     学校要求所设计的 \cls{毕业论文格式 tex04 版},以及 2008 年
%     张越同学修改之后的 \cls{毕业论文格式 tex08 版},这是专为
%     数院本科生撰写毕业论文而设计的
%     \scite{数院毕业论文格式,数院毕业论文格式更新};
%   \item Pandoxie 编写的 \cls{FDU-Thesis-Latex}
%     \scite{pandoxie2014fduthesislatex},基本满足了博士(硕士)
%     毕业论文格式要求,使用人数较多;
%   \item richarddzh 编写的硕士论文模板 \cls{fudan-thesis}
%     \scite{richard2016fudanthesis}。
% \end{itemize}
% 以上这些模板大都没有经过系统的设计,也鲜有后续维护。相比之下,
% 清华大学 \scite{thuthesis}、重庆大学 \scite{cquthesis}、
% 中国科学技术大学 \scite{ustcthesis} 中国科学院大学 \scite{ucasthesis}
% 以及友校上海交通大学 \scite{sjtuthesis}等,都有成熟、
% 稳定的解决方案,值得参考。
%
% 本模板将借鉴前辈经验,重新设计,并使用 \LaTeX3
% \scite{source3} 编写,以适应 \TeX{} 技术发展潮流;
% 同时还将构建一套简洁的接口,方便用户使用。
%^^A! \cls{fduthesis} is a thesis template for Fudan University.
%^^A! This template is mostly written in \LaTeX3 syntax, and
%^^A! provides a simple interface for users.
%^^A!
%
% \subsection*{\LaTeX{} 入门}
%^^A! \subsection*{Getting started with \LaTeX{}}
%^^A!
%
% 本文档并非是一份 \LaTeX{} 零基础教程。如果您是完完全全的新手,
% 建议先阅读相关入门文档,如刘海洋编著的《\LaTeX{} 入门》
% \scite{刘海洋2013latex入门} 第一章,或大名鼎鼎的“\pkg{lshort}”
% \scite{lshort} 及其中文翻译版 \scite{lshort-zh-cn}。当然,
% 网络上的入门教程多如牛毛,您可以自行选取。
%^^A! This documentation is \emph{not} a \LaTeX{} tutorial at
%^^A! starter's level. If you are totally a newbie, please read some
%^^A! introductions like the famous \pkg{lshort}. Of course, there
%^^A! are countless \LaTeX{} tutorials on the Internet. You can
%^^A! choose whatever you like.
%^^A!
%
% \subsection*{关于本文档}
%^^A! \subsection*{About this documentation}
%^^A!
%
% 本文采用不同字体表示不同内容。无衬线字体表示宏包名称,如
% \pkg{xeCJK} 宏包、\cls{fduthesis} 文档类等;等宽字体表示代码或
% 文件名,如 \cs{fdusetup} 命令、\env{abstract} 环境、\TeX{} 文档
% \file{thesis.tex} 等;带有尖括号的楷体(或西文斜体)表示命令参数,
% 如 \meta{模板选项}、\meta{English title} 等。在使用时,参数两侧
% 的尖括号不必输入。示例代码进行了语法高亮处理,以方便阅读。
%^^A! In this documentation, different typefaces are used to
%^^A! represent different contents. Packages and classes are shown
%^^A! in sans-serif font, e.g.\ \pkg{xeCJK} package and
%^^A! \cls{fduthesis} class. Commands and file names are shown in
%^^A! monospaced font, e.g.\ command \cs{fdusetup}, environment
%^^A! \env{abstract} and \TeX{} document \file{thesis.tex}.
%^^A! Italic-shaped font with angle brackets outside means arguments,
%^^A! e.g.\ \meta{English title}. However, you do not need to type
%^^A! the brackets when using these commands. The example code has
%^^A! proper syntax highlighting so it will be much easier to read.
%^^A!
%
% 在用户手册中,带有蓝色侧边线的为 \LaTeX{} 代码,而带有粉色侧边线
% 的则为命令行代码,请注意区分。模板提供的选项、命令、环境等,
% 均用横线框起,同时给出使用语法和相关说明。
%^^A! \LaTeX{} code lines will have a blue line on their left, while
%^^A! for command lines there will be a pink line. The options,
%^^A! commands and environments in \cls{fduthesis} will be surrounded
%^^A! by two horizontal lines. Their usages and descriptions are
%^^A! provided at the same time.
%^^A!
%
% 本模板中的选项、命令或环境可以分为以下三类:
% \begin{itemize}
%   \item 名字后面带有 \rexptarget\rexpstar{} 的,表示只能在^^A
%     \emph{中文模板}中使用;
%   \item 名字后面带有 \exptarget\expstar{} 的,表示只能在^^A
%     \emph{英文模板}中使用;
%   \item 名字后面不带有特殊符号的,表示既可以在中文模板中使用,
%     也可以在英文模板中使用。
% \end{itemize}
%^^A! The options, commands and environments in \cls{fduthesis} can be
%^^A! divided into the following three types:
%^^A! \begin{itemize}
%^^A!   \item Those can be only used in \emph{Chinese templates} are
%^^A!     indicated by \rexptarget\rexpstar{}.
%^^A!   \item Those can be only used in \emph{English templates} are
%^^A!     indicated by \rexptarget\expstar{}.
%^^A!   \item If they do not have special characters afterwards, then
%^^A!     you can use them in both Chinese and English templates.
%^^A! \end{itemize}
%^^A!
%
% 代码实现主要面向对 \LaTeX{} 宏包开发感兴趣的用户。如果您有任何改进
% 意见或者功能需求,欢迎前往 GitHub 仓库
% \href{https://github.com/stone-zeng/fduthesis/issues}{提交 issue}。
%^^A! If you want to read the implementation part, please turn to the
%^^A! Chinese version. Any issues or feature requests can be submitted
%^^A! in the \href{https://github.com/stone-zeng/fduthesis/issues}%
%^^A! {GitHub repository}.
%^^A!
%
% 文档的最后还提供了版本历史和代码索引,以供查阅。
%
% \section{安装}
%^^A! \section{Installation}
%^^A!
%
% \subsection{获取 \cls{fduthesis}}
%^^A! \subsection{Obtaining \cls{fduthesis}}
%^^A!
%
% \subsubsection{标准安装}
%^^A! \subsubsection{Standard installation}
%^^A!
%
% 如果没有特殊理由,始终建议您使用宏包管理器安装 \cls{fduthesis}。
% 例如在 \TeXLive{} 中,执行(可能需要管理员权限)
% \begin{shellexample}[morekeywords={tlmgr,install}]
%   tlmgr install fduthesis
% \end{shellexample}
% 即可完成安装。
%^^A! If there are no special reasons, it is always recommended to
%^^A! install \cls{fduthesis} with a package manager. For example,
%^^A! the following command will install the package in \TeXLive{}
%^^A! (administrator permission may be required):
%^^A! \begin{shellexample}[morekeywords={tlmgr,install}]
%^^A!   tlmgr install fduthesis
%^^A! \end{shellexample}
%^^A!
%
% 在 \TeXLive{} 和 \MiKTeX{} 中,您还可以通过图形界面进行安装,
% 此处不再赘述。
%^^A! In \TeXLive{} and \MiKTeX{}, you can also install \cls{fduthesis}
%^^A! through a graphical interface. It's rather simple and will not be
%^^A! described here.
%^^A!
%
% \subsubsection{手动安装}
%^^A! \subsubsection{Install manually}
%^^A!
%
% 如果您需要从 CTAN 上自行下载并手动安装,较好的方法是使用 TDS
% 安装包:
% \begin{itemize}
%   \item 从 CTAN 上下载 \cls{fduthesis} 的
%     \href{http://mirror.ctan.org/install/macros/latex/contrib/fduthesis.tds.zip}{TDS 安装包};
%   \item 按目录结构将 \file{fduthesis.tds.zip} 中的文件复制到 \TeX{}
%     发行版的本地 TDS 根目录;
%   \item 执行 \bashcmd{mktexlsr} 刷新文件名数据库以完成安装。
% \end{itemize}
%^^A! If you want to download the template from CTAN and install it
%^^A! manually, the recommended way is to use the TDS ZIP file:
%^^A! \begin{itemize}
%^^A!   \item Download the \href{http://mirror.ctan.org/install/macros/latex/contrib/fduthesis.tds.zip}%
%^^A!     {TDS ZIP file} for \cls{fduthesis};
%^^A!   \item Copy all the files in \file{fduthesis.tds.zip} into the
%^^A!     local TDS directory of \TeX{} distribution.
%^^A!   \item Run \bashcmd{mktexlsr} to update the ls-R database.
%^^A! \end{itemize}
%^^A
% 您也可以从源代码直接生成模板(不推荐):
% \begin{itemize}
%   \item 打开 \href{https://github.com/stone-zeng/fduthesis}^^A
%     {项目主页},点击“Clone or download”,并选择“Download ZIP”,
%     下载 \file{fduthesis-master.zip};如果您的电脑中安装有 git
%     程序,也可通过以下命令直接克隆代码仓库:
%     \begin{shellexample}[gobble=7,alsoletter={.},^^A
%         morekeywords={git,clone}]
%       git clone https://github.com/stone-zeng/fduthesis.git
%     \end{shellexample}
%   \item 解压并进入到 \file{source} 文件夹,执行以下命令以生成
%     模板的各组件:
%     \begin{shellexample}[gobble=7,morekeywords={xetex}]
%       xetex fduthesis.dtx
%     \end{shellexample}
%   \item 将生成的文档类(\file{.cls})、宏包(\file{.sty})以及
%     参数配置文件(\file{.def})复制到 \TeX{} 发行版本地 TDS 树
%     的 \path{texmf-local/tex/latex/fduthesis/} 目录下,并执行
%     \bashcmd{mktexlsr} 刷新文件名数据库,方可完成安装。
%   \item 使用 \cls{fduthesis} 撰写论文时,您还需要从代码仓库下的
%     \file{testfiles/support} 目录中复制 \file{fudan-name.pdf}
%     文件至工作目录,以确保封面中的校名图片可以正确显示。
% \end{itemize}
%^^A! Although not recommended, you may generate the whole template
%^^A! from source code as well:
%^^A! \begin{itemize}
%^^A!   \item Open the project's \href{https://github.com/stone-zeng/fduthesis}%
%^^A!     {homepage}, click ``Clone or download'' and choose
%^^A!     ``Download ZIP'' to download \file{fduthesis-master.zip}.
%^^A!     If you have git program on your computer, you can also
%^^A!     clone the repository directly:
%^^A!     \begin{shellexample}[gobble=5,alsoletter={.},%
%^^A!         morekeywords={git,clone}]
%^^A!       git clone https://github.com/stone-zeng/fduthesis.git
%^^A!     \end{shellexample}
%^^A!   \item Extract files, and get into the \file{source} directory.
%^^A!     Run the following command to generate all the components:
%^^A!     \begin{shellexample}[gobble=5,morekeywords={xetex}]
%^^A!       xetex fduthesis.dtx
%^^A!     \end{shellexample}
%^^A!   \item Copy the generated document classes (\file{.cls}),
%^^A!     packages (\file{.sty}) and configuration files (\file{.def})
%^^A!     into \path{texmf-local/tex/latex/fduthesis/} under the local
%^^A!     TDS tree of \TeX{} distribution. Then run \bashcmd{mktexlsr}
%^^A!     to update the ls-R database.
%^^A!   \item When writing your thesis with \cls{fduthesis}, you need
%^^A!     to copy the file \file{fudan-name.pdf} (can be found in the
%^^A!     \file{testfiles/support} directory of the Git repository) to
%^^A!     your working directory, to make sure that the logo in the
%^^A!     cover can be displayed correctly.
%^^A! \end{itemize}
%^^A!
%
% \subsubsection{扁平化安装}
%^^A! \subsubsection{\cls{fduthesis} on the fly}
%^^A!
%
% 如果您不希望安装本模板,但需要立刻使用,也可以使用模板提供的安装脚本。
% 从 GitHub 上获取代码仓库后,执行 \file{install-win.bat}(Windows 系统)
% 或 \file{install-linux.sh}(Linux 系统),所有需要的文件便会在
% \file{thesis} 文件夹中生成。
%^^A! If you don't want to install \cls{fduthesis} but need to use it
%^^A! at once, you can try the installation scripts. Download the
%^^A! repository from GitHub, run \file{install-win.bat} (on Windows)
%^^A! or \file{install-linux.sh} (on Linux), then all the necessary
%^^A! files will be found in the \file{thesis} folder.
%^^A!
%
% \subsection{模板组成}
%^^A! \subsection{Composition of the template}
%^^A!
%
% 本模板主要包含核心文档类、配置文件、附属宏包以及用户文档等几个
% 部分,其具体组成见表~\ref{tab:fduthesis-components}。
%^^A! There are several parts in \cls{fduthesis}, including kernel
%^^A! template classes, configuration files, affiliated packages and
%^^A! user's guides. More details are listed in table~%
%^^A! \ref{tab:fduthesis-components}.
%^^A!
%
% \begin{table}[ht]
%   \caption{\cls{fduthesis} 的主要组成部分}
%   \label{tab:fduthesis-components}
%   \centering
%   \begin{tabular}{lp{20em}}
%     \toprule
%     \textbf{文件} & \textbf{功能说明} \\
%     \midrule
%     \file{fduthesis.cls}          & 中文模板文档类 \\
%     \file{fduthesis-en.cls}       & 英文模板文档类 \\
%     \file{fduthesis.def}          & 参数配置文件,用于设定
%       \cls{fduthesis} 的初始参数,不建议您自行改动 \\
%     \file{fdudoc.cls}             & 用户手册文档类 \\
%     \file{fdulogo.sty}            & 复旦大学视觉识别系统 \\
%     \file{fudan-emblem.pdf}       & 校徽 \\
%     \file{fudan-emblem-new.pdf}   & 校徽(重修版) \\
%     \file{fudan-name.pdf}         & 校名图片 \\
%     \file{README.md}              & 简要自述 \\
%     \ifdefined\FDUCODEDOC
%       \file{fduthesis.pdf}        & 中文用户手册 \\
%       \file{fduthesis-en.pdf}     & 英文用户手册 \\
%       \file{fduthesis-code.pdf}   & 模板实现代码(本文档) \\
%     \else
%       \file{fduthesis.pdf}        & 中文用户手册(本文档) \\
%       \file{fduthesis-en.pdf}     & 英文用户手册 \\
%       \file{fduthesis-code.pdf}   & 模板实现代码 \\
%     \fi
%     \file{fduthesis-template.tex} & 空白模板,可据此为基础完成论文
%       撰写 \\
%     \bottomrule
%   \end{tabular}
% \end{table}
%^^A! \begin{table}[ht]
%^^A!   \caption{The main components of \cls{fduthesis}}
%^^A!   \label{tab:fduthesis-components}
%^^A!   \centering
%^^A!   \begin{tabular}{lp{24em}}
%^^A!     \toprule
%^^A!     \textbf{Files} & \textbf{Descriptions} \\
%^^A!     \midrule
%^^A!     \file{fduthesis.cls}          & Document class for Chinese thesis. \\
%^^A!     \file{fduthesis-en.cls}       & Document class for English thesis.\\
%^^A!     \file{fduthesis.def}          & Configuration parameters file
%^^A!       for \cls{fduthesis}. Please do \emph{not} modify it. \\
%^^A!     \file{fdudoc.cls}             & Document class for user guides. \\
%^^A!     \file{fdulogo.sty}            & Fudan University's visual identity. \\
%^^A!     \file{fudan-emblem.pdf}       & University emblem. \\
%^^A!     \file{fudan-emblem-new.pdf}   & University emblem (revised version). \\
%^^A!     \file{fudan-name.pdf}         & Figure of university name. \\
%^^A!     \file{README.md}              & The brief introduction. \\
%^^A!     \file{fduthesis.pdf}          & User's guide in Chinese. \\
%^^A!     \file{fduthesis-en.pdf}       & User's guide in English (this
%^^A!       document). \\
%^^A!     \file{fduthesis-code.pdf}     & Code implementation. \\
%^^A!     \file{fduthesis-template.tex} & An empty thesis template, and you can
%^^A!       write your thesis based on it. \\
%^^A!     \bottomrule
%^^A!   \end{tabular}
%^^A! \end{table}
%^^A!
%
% \section{使用说明}
%^^A! \section{User's guide}
%^^A!
%
% \subsection{基本用法}
%^^A! \subsection{Getting started}
%^^A!
%
% 以下是一份简单的 \TeX{} 文档,它演示了 \cls{fduthesis}
% 的最基本用法:
%^^A! Here is a minimal \TeX{} file for \cls{fduthesis}:
%^^A+
% \begin{latexexample}[deletetexcs={\documentclass},%
%     moretexcs={\chapter},morekeywords={\documentclass},%
%     emph={[2]document}]
%   % thesis.tex
%   \documentclass{fduthesis}
%   \begin{document}
%     \chapter{欢迎}
%     \section{Welcome to fduthesis!}
%     你好,\LaTeX{}!
%   \end{document}
% \end{latexexample}
%^^A-
%^^A!
%
% 按照 \ref{subsec:编译方式}~小节中的方式编译该文档,您应当得到
% 一篇 5 页的文章。当然,这篇文章的绝大部分都是空白的。
%^^A! Compile this file under the instructions in subsection~%
%^^A! \ref{subsec:compilation}, you will get a 5-page article.
%^^A! Of course, most of it will be blank, as you may predicate.
%^^A!
%
% 英文模板可以用类似的方式使用:
%^^A! The English version can be used in the same way:
%^^A+
% \begin{latexexample}[deletetexcs={\documentclass},%
%     moretexcs={\chapter},morekeywords={\documentclass},%
%     emph={[2]document}]
%   % thesis-en.tex
%   \documentclass{fduthesis-en}
%   \begin{document}
%     \chapter{Welcome}
%     \section{Welcome to fduthesis!}
%     Hello, \LaTeX{}!
%   \end{document}
% \end{latexexample}
%^^A-
% 英文模板只对正文部分进行了改动,封面、指导小组成员以及声明页仍将
% 显示为中文。
%^^A! The differences between English and Chinese version only
%^^A! live in the main body. Thesis cover, instructors list and
%^^A! declaration page are still printed in Chinese.
%^^A!
%
% \subsection{编译方式} \label{subsec:编译方式}
%^^A! \subsection{Compilation} \label{subsec:compilation}
%^^A!
%
% 本模板不支持 \pdfTeX{} 引擎,请使用 \XeLaTeX{} 或 \LuaLaTeX{}
% 编译。推荐使用 \XeLaTeX{}。为了生成正确的目录、脚注以及交叉引用,
% 您至少需要连续编译两次。
%^^A! \cls{fduthesis} does NOT support \pdfTeX{}. Please use
%^^A! \XeLaTeX{} or \LuaLaTeX{} to compile, and \XeLaTeX{} is
%^^A! recommended. To get the correct table of contents, footnotes
%^^A! and cross-references, you need to compile the source file at
%^^A! least twice.
%^^A!
%
% 以下代码中,假设您的 \TeX{} 源文件名为 \file{thesis.tex}。
% 使用 \XeLaTeX{} 编译论文,请在命令行中执行
% \begin{shellexample}[morekeywords={xelatex}]
%   xelatex thesis
%   xelatex thesis
% \end{shellexample}
% 或使用 \pkg{latexmk}:
% \begin{shellexample}[morekeywords={latexmk},emph={-xelatex}]
%   latexmk -xelatex thesis
% \end{shellexample}
%^^A! In the following example, suppose your \TeX{} source file is
%^^A! \file{thesis.tex}. Please execute the following commands if
%^^A! you want to use \XeLaTeX{}:
%^^A! \begin{shellexample}[morekeywords={xelatex}]
%^^A!   xelatex thesis
%^^A!   xelatex thesis
%^^A! \end{shellexample}
%^^A! You can use \pkg{latexmk} as well:
%^^A! \begin{shellexample}[morekeywords={latexmk},emph={-xelatex}]
%^^A!   latexmk -xelatex thesis
%^^A! \end{shellexample}
%^^A!
%
% 使用 \LuaLaTeX{} 编译论文,请在命令行中执行
% \begin{shellexample}[morekeywords={lualatex}]
%   lualatex thesis
%   lualatex thesis
% \end{shellexample}
% 或者
% \begin{shellexample}[morekeywords={latexmk},emph={-lualatex}]
%   latexmk -lualatex thesis
% \end{shellexample}
%^^A! \LuaLaTeX{} can be used in a similar way:
%^^A! \begin{shellexample}[morekeywords={lualatex}]
%^^A!   lualatex thesis
%^^A!   lualatex thesis
%^^A! \end{shellexample}
%^^A! or
%^^A! \begin{shellexample}[morekeywords={latexmk},emph={-lualatex}]
%^^A!   latexmk -lualatex thesis
%^^A! \end{shellexample}
%^^A!
%
% \subsection{模板选项}
%^^A! \subsection{Options of the template}
%^^A!
%
% 所谓“模板选项”,指需要在引入文档类的时候指定的选项:
% \begin{latexexample}[deletetexcs={\documentclass},%
%     morekeywords={\documentclass}]
%   \documentclass(*\oarg{模板选项}*){fduthesis}
%   \documentclass(*\oarg{模板选项}*){fduthesis-en}
% \end{latexexample}
%^^A! You can specify some \emph{template options} when loading
%^^A! \cls{fduthesis}:
%^^A! \begin{latexexample}[deletetexcs={\documentclass},%
%^^A!     morekeywords={\documentclass}]
%^^A!   \documentclass(*\oarg{options}*){fduthesis}
%^^A!   \documentclass(*\oarg{options}*){fduthesis-en}
%^^A! \end{latexexample}
%^^A!
%
% 有些模板选项为布尔型,它们只能在 \opt{true} 和 \opt{false}
% 中取值。对于这些选项,\kvopt{\meta{选项}}{true} 中的“|= true|”
% 可以省略。
%^^A! Some options are \emph{boolean} --- they only take the value
%^^A! \opt{true} or \opt{false}. For these options, you can
%^^A! abbreviate ``\kvopt{\meta{option}}{true}'' simply to
%^^A! ``\opt{\meta{option}}''.
%^^A!
%
%^^A+
% \begin{function}[added=2018-02-01]{type}
%   \begin{fdusyntax}[emph={[1]type}]
%     type = (*<doctor|master|(bachelor)>*)
%   \end{fdusyntax}
%^^A-
%   选择论文类型。三种选项分别代表博士学位论文、硕士学位论文和本科
%   毕业论文。
% \end{function}
%^^A!   Choose the type of your thesis. The three options represent
%^^A!   doctoral dissertation, master degree thesis and undergraduate
%^^A!   thesis, respectively.
%^^A! \end{function}
%^^A!
%
% \begin{function}{oneside,twoside}
%   指明论文的单双面模式,默认为 \opt{twoside}。该选项会影响每章
%   的开始位置,还会影响页眉样式。
% \end{function}
%^^A! \begin{function}{oneside,twoside}
%^^A!   Specify whether single or double sided output should be
%^^A!   generated. \opt{twoside} will be chosen by default. These
%^^A!   option will determine where the new chapters begin and how
%^^A!   the headers display. The option \opt{twoside} does
%^^A!   \emph{not} tell the printer to actually make a two-sided
%^^A!   printout.
%^^A! \end{function}
%^^A!
%
% 在双面模式(\opt{twoside})下,按照通常的排版惯例,每章应只从
% 奇数页(在右)开始;而在单页模式(\opt{oneside})下,则可以从
% 任意页面开始。本模板中,目录、摘要、符号表等均视作章,也按相同
% 方式排版。
%^^A! If choosing \opt{twoside}, chapters will begin at the odd pages
%^^A! (right hand). However, they will begin at arbitrary pages
%^^A! available when choosing \opt{oneside}. Table of contents,
%^^A! abstract and the list of symbols are considered as chapters and
%^^A! processed in the same way.
%^^A!
%
% 双面模式下,正文部分偶数页(在左)的左页眉显示章标题,奇数页
% (在右)的右页眉显示节标题;前置部分的页眉按同样格式显示,但文字
% 均为对应标题(如“目录”、“摘要”等)。
% 而在单面模式下,正文部分则页面不分奇偶,均同时显示左、右页眉,
% 文字分别为章标题和节标题;前置部分只有中间页眉,显示对应标题。
%^^A! At two-sided mode, left headers on the even pages (left hand)
%^^A! in \emph{main body} will show the title of chapters, while the
%^^A! right headers on the odd pages (right hand) will show the
%^^A! title of sections. Headers in \emph{front matter} have the
%^^A! same style, but they will only show the title as ``Contents'',
%^^A! ``Abstract'', etc.
%^^A!
%^^A! At one-sided mode, both left and right headers on \emph{all}
%^^A! pages in main body will be shown. The text is the title of
%^^A! chapters and sections, respectively. In front matter, there
%^^A! are only middle headers, which show the corresponding titles.
%^^A!
%
% \begin{function}{draft}
%   \begin{fdusyntax}[emph={[1]draft}]
%     draft = (*<\TFF>*)
%   \end{fdusyntax}
%   选择是否开启草稿模式,默认关闭。
% \end{function}
%^^A! \begin{function}{draft}
%^^A!   \begin{fdusyntax}[emph={[1]draft}]
%^^A!     draft = (*<\TFF>*)
%^^A!   \end{fdusyntax}
%^^A!   Enable draft mode. Default off.
%^^A! \end{function}
%^^A!
%
% 草稿模式为全局选项,会影响到很多宏包的工作方式。
% 开启之后,主要的变化有:
% \begin{itemize}
%   \item 把行溢出的盒子显示为黑色方块;
%   \item 不实际插入图片,只输出一个占位方框;
%   \item 关闭超链接渲染,也不再生成 PDF 书签;
%   \item 显示页面边框。
% \end{itemize}
%^^A! \opt{draft} is a global option and will affect many packages.
%^^A! You may notice the following changes when using \opt{draft}:
%^^A! \begin{itemize}
%^^A!   \item Lines with overfull \tn{hbox}'s will be marked with
%^^A!     a thick black square on the right margin.
%^^A!   \item Will not include graphics files actually, but instead
%^^A!     print a box of the size the graphic would take up, as well
%^^A!     as the file name.
%^^A!   \item Will not make hyperlinks and PDF bookmarks.
%^^A!   \item Show the page frames.
%^^A! \end{itemize}
%^^A!
%
% \begin{function}[added=2018-01-31]{config}
%   \begin{fdusyntax}[emph={[1]config}]
%     config = (*\marg{文件}*)
%   \end{fdusyntax}
%   用户配置文件的文件名。默认为空,即不载入用户配置文件。
% \end{function}
%^^A! \begin{function}[added=2018-01-31]{config}
%^^A!   \begin{fdusyntax}[emph={[1]config}]
%^^A!     config = (*\marg{file}*)
%^^A!   \end{fdusyntax}
%^^A!   File name of user profile. Default value is empty, so no
%^^A!   profile is loaded automatically.
%^^A! \end{function}
%^^A!
%
% \subsection{参数设置}
%^^A! \subsection{More options}
%^^A!
%
% \begin{function}{\fdusetup}
%   \begin{fdusyntax}[morekeywords={\fdusetup}]
%     \fdusetup(*\marg{键值列表}*)
%   \end{fdusyntax}
%   本模板提供了一系列选项,可由您自行配置。载入文档类之后,以下
%   所有选项均可通过统一的命令 \cs{fdusetup} 来设置。
% \end{function}
%^^A! \begin{function}{\fdusetup}
%^^A!   \begin{fdusyntax}[morekeywords={\fdusetup}]
%^^A!     \fdusetup(*\marg{key-value list}*)
%^^A!   \end{fdusyntax}
%^^A!   \cls{fduthesis} has provided a number of options, which
%^^A!   can be given via the general command \cs{fdusetup}.
%^^A! \end{function}
%^^A!
%
% \cs{fdusetup} 的参数是一组由(英文)逗号隔开的选项列表,列表中的
% 选项通常是 \kvopt{\meta{key}}{\meta{value}} 的形式。部分选项的
% \meta{value} 可以省略。对于同一项,后面的设置将会覆盖前面的设置。
% 在下文的说明中,将用\textbf{粗体}表示默认值。
%^^A! The argument of \cs{fdusetup} is a set of comma-separated option
%^^A! list. The options usually have the form of \kvopt{\meta{key}}%
%^^A! {\meta{value}} and in some cases \meta{value} can be omitted.
%^^A! For the same option, the values given later will override the
%^^A! the previous ones. Default values are indicated in
%^^A! \textbf{boldface} in the following descriptions.
%^^A!
%
% \cs{fdusetup} 采用 \LaTeX3 风格的键值设置,支持不同类型以及多种
% 层次的选项设定。键值列表中,“|=|”左右的空格不影响设置;但需注意,
% 参数列表中不可以出现空行。
%^^A! \cs{fdusetup} follows \LaTeX3 key-value style, and different
%^^A! types as well as various levels options are supported. In the
%^^A! key-value list, spaces around ``|=|'' will be trimmed; however,
%^^A! blank lines should never appear in the argument.
%^^A!
%
% 与模板选项相同,布尔型的参数可以省略 \kvopt{\meta{选项}}{true}
% 中的“|= true|”。
%^^A! Similar with template options, ``\kvopt{\meta{option}}{true}''
%^^A! can be abbreviated to \opt{\meta{option}} for boolean type.
%^^A!
%
% 另有一些选项包含子选项,如 \opt{style} 和 \opt{info} 等。它们可以
% 按如下两种等价方式来设定:
%^^A! Some options, such as \opt{style} and \opt{info}, may have
%^^A! sub-options. They can be set by the following two equivalent
%^^A! methods:
%^^A+
% \begin{latexexample}[morekeywords={\fdusetup},%
%     emph={[1]style,cjk-font,font-size,info,title,title*,author,author*,department}]
%   \fdusetup{
%     style = {cjk-font = adobe, font-size = -4},
%     info  = {
%       title      = {论动体的电动力学},
%       title*     = {On the Electrodynamics of Moving Bodies},
%       author     = {阿尔伯特·爱因斯坦},
%       author*    = {Albert Einstein},
%       department = {物理学系}
%     }
%   }
% \end{latexexample}
%^^A-
% 或者
%^^A! or
%^^A+
% \begin{latexexample}[morekeywords={\fdusetup},%
%     emph={[1]style,cjk-font,font-size,info,title,title*,author,author*,department}]
%   \fdusetup{
%     style/cjk-font  = adobe,
%     style/font-size = -4,
%     info/title      = {论动体的电动力学},
%     info/title*     = {On the Electrodynamics of Moving Bodies},
%     info/author     = {阿尔伯特·爱因斯坦},
%     info/author*    = {Albert Einstein},
%     info/department = {物理学系}
%   }
% \end{latexexample}
%^^A-
%^^A!
%
% 注意 “|/|” 的前后均不可以出现空白字符。
%^^A! Note that you may \emph{not} put spaces around ``|/|''.
%^^A!
%
% \subsubsection{论文格式} \label{subsubsec:论文格式}
%^^A! \subsubsection{Style and format} \label{subsubsec:style-and-format}
%^^A!
%
% \begin{function}{style}
%   \begin{fdusyntax}[emph={[1]style}]
%     style = (*\marg{键值列表}*)
%     style/(*\meta{key}*) = (*\meta{value}*)
%   \end{fdusyntax}
%   该选项包含许多子项目,用于设置论文格式。具体内容见下。
% \end{function}
%^^A! \begin{function}{style}
%^^A!   \begin{fdusyntax}[emph={[1]style}]
%^^A!     style = (*\marg{key-value list}*)
%^^A!     style/(*\meta{key}*) = (*\meta{value}*)
%^^A!   \end{fdusyntax}
%^^A!   This general option is for setting the thesis style and format.
%^^A!   See the following details.
%^^A! \end{function}
%^^A!
%
%^^A+
% \begin{function}[updated=2019-03-05]{style/font}
%   \begin{fdusyntax}[emph={[1]font}]
%     font = (*<garamond|libertinus|lm|palatino|(times)|times*|none>*)
%   \end{fdusyntax}
%^^A-
%   设置西文字体(包括数学字体)。具体配置见表~\ref{tab:font}。
% \end{function}
%^^A!   Set fonts (including math fonts). The details can be found in table~\ref{tab:font}.
%^^A! \end{function}
%^^A!
%
% \begin{table}[ht]
% \begin{threeparttable}
%   \caption{西文字体配置}
%   \label{tab:font}
%   \centering
%   \begin{tabular}{ccccc}
%     \toprule
%       & \strong{正文字体} & \strong{无衬线字体} & \strong{等宽字体} & \strong{数学字体} \\
%     \midrule
%       |garamond|        & EB Garamond         & Libertinus Sans & LM Mono\tnote{a} & Garamond Math   \\
%       |libertinus|      & Libertinus Serif    & Libertinus Sans & LM Mono          & Libertinus Math \\
%       |lm|              & LM Roman            & LM Sans         & LM Mono          & LM Math         \\
%       |palatino|        & TG Pagella\tnote{b} & Libertinus Sans & LM Mono          & TG Pagella Math \\
%       |times|           & XITS                & TG Heros        & TG Cursor        & XITS Math       \\
%       |times*|\tnote{c} & Times New Roman     & Arial           & Courier New      & XITS Math       \\
%     \bottomrule
%   \end{tabular}
%   \begin{tablenotes}
%     \item[a] “LM”是 Latin Modern 的缩写。
%     \item[b] “TG”是 TeX Gyre 的缩写。
%     \item[c] 本行中,Times New Roman、Arial 和 Courier New 是商业字体,
%       在 Windows 和 macOS 系统上均默认安装。
%   \end{tablenotes}
% \end{threeparttable}
% \end{table}
%^^A! \begin{table}[ht]
%^^A! \begin{threeparttable}
%^^A!   \caption{Font configuration}
%^^A!   \label{tab:font}
%^^A!   \centering
%^^A!   \begin{tabular}{ccccc}
%^^A!     \toprule
%^^A!       & \strong{Roman} & \strong{Sans-serif} & \strong{Monospaced} & \strong{Math} \\
%^^A!     \midrule
%^^A!       |garamond|        & EB Garamond         & Libertinus Sans & LM Mono\tnote{a} & Garamond Math   \\
%^^A!       |libertinus|      & Libertinus Serif    & Libertinus Sans & LM Mono          & Libertinus Math \\
%^^A!       |lm|              & LM Roman            & LM Sans         & LM Mono          & LM Math         \\
%^^A!       |palatino|        & TG Pagella\tnote{b} & Libertinus Sans & LM Mono          & TG Pagella Math \\
%^^A!       |times|           & XITS                & TG Heros        & TG Cursor        & XITS Math       \\
%^^A!       |times*|\tnote{c} & Times New Roman     & Arial           & Courier New      & XITS Math       \\
%^^A!     \bottomrule
%^^A!   \end{tabular}
%^^A!   \begin{tablenotes}
%^^A!     \item[a] ``LM'' is the abbreviation of Latin Modern.
%^^A!     \item[b] ``TG'' is the abbreviation of TeX Gyre.
%^^A!     \item[c] Here, Times New Roman, Arial and Courier New are commercial fonts. They are
%^^A!       installed on Windows and macOS by default.
%^^A!   \end{tablenotes}
%^^A! \end{threeparttable}
%^^A! \end{table}
%
%^^A+
% \begin{function}[rEXP,updated=2019-03-05]{style/cjk-font}
%   \begin{fdusyntax}[emph={[1]cjk-font}]
%     cjk-font = (*<adobe|(fandol)|founder|mac|sinotype|sourcehan|windows|none>*)
%   \end{fdusyntax}
%^^A-
%   设置中文字体。具体配置见表~\ref{tab:cjk-font}。
% \end{function}
%^^A!   Set CJK (Chinese, Japanese and Korean) fonts. The details can be found in
%^^A!   table~\ref{tab:cjk-font}.
%^^A! \end{function}
%^^A!
%
% \begin{table}[ht]
%   \caption{中文字体配置}
%   \label{tab:cjk-font}
%   \centering
%   \begin{tabular}{cccc}
%     \toprule
%       & \strong{正文字体(宋体)} & \strong{无衬线字体(黑体)} & \strong{等宽字体(仿宋)} \\
%     \midrule
%       \multirow{2}*{|adobe|}     & Adobe 宋体          & Adobe  黑体        & Adobe  仿宋        \\
%                                  & Adobe Song Std      & Adobe Heiti Std    & Adobe Fangsong Std \\
%       \multirow{2}*{|fandol|}    & Fandol 宋体         & Fandol 黑体        & Fandol 仿宋        \\
%                                  & FandolSong          & FandolHei          & FandolFang         \\
%       \multirow{2}*{|founder|}   & 方正书宋            & 方正黑体           & 方正仿宋           \\
%                                  & FZShuSong-Z01       & FZHei-B01          & FZFangSong-Z02     \\
%       \multirow{2}*{|mac|}       & (华文)宋体-简     & (华文)黑体-简    & 华文仿宋           \\
%                                  & Songti SC           & Heiti SC           & STFangsong         \\
%       \multirow{2}*{|sinotype|}  & 华文宋体            & 华文黑体           & 华文仿宋           \\
%                                  & STSong              & STHeiti            & STFangsong         \\
%       \multirow{2}*{|sourcehan|} & 思源宋体            & 思源黑体           & ---                \\
%                                  & Source Han Serif SC & Source Han Sans SC & ---                \\
%       \multirow{2}*{|windows|}   & (中易)宋体        & (中易)黑体       & (中易)仿宋       \\
%                                  & SimSun              & SimHei             & FangSong           \\
%     \bottomrule
%   \end{tabular}
% \end{table}
%^^A! \begin{table}[ht]
%^^A!   \caption{CJK font configuration}
%^^A!   \label{tab:cjk-font}
%^^A!   \centering
%^^A!   \begin{tabular}{cccc}
%^^A!     \toprule
%^^A!       & \strong{Roman (song)} & \strong{Sans-serif (hei)} & \strong{Monospaced (fang)} \\
%^^A!     \midrule
%^^A!       |adobe|     & Adobe Song Std      & Adobe Heiti Std    & Adobe Fangsong Std \\
%^^A!       |fandol|    & FandolSong          & FandolHei          & FandolFang         \\
%^^A!       |founder|   & FZShuSong-Z01       & FZHei-B01          & FZFangSong-Z02     \\
%^^A!       |mac|       & Songti SC           & Heiti SC           & STFangsong         \\
%^^A!       |sinotype|  & STSong              & STHeiti            & STFangsong         \\
%^^A!       |sourcehan| & Source Han Serif SC & Source Han Sans SC & ---                \\
%^^A!       |windows|   & SimSun              & SimHei             & FangSong           \\
%^^A!     \bottomrule
%^^A!   \end{tabular}
%^^A! \end{table}
%^^A!
%
% 启用 \kvopt{font}{none} 或 \kvopt{cjk-font}{none} 之后,模板将关闭
% 默认西文 / 中文字体设置。此时,您需要自行使用 \cs{setmainfont}、
% \cs{setCJKmainfont}、\cs{setmathfont} 等命令来配置字体。
%^^A! When you choose \kvopt{font}{none} or \kvopt{cjk-font}{none},
%^^A! \cls{fduthesis} will disable the default western/CJK font
%^^A! settings. You may use \cs{setmainfont}, \cs{setCJKmainfont}
%^^A! and \cs{set\-math\-font}, etc.\ to configure the fonts manually.
%^^A!
%
%^^A+
% \begin{function}{style/font-size}
%   \begin{fdusyntax}[emph={[1]font-size}]
%     font-size = (*<(-4)|5>*)
%   \end{fdusyntax}
%^^A-
%   设置论文的基础字号。
% \end{function}
%^^A!   Specify the basic font size in your thesis.
%^^A! \end{function}
%^^A!
%
%^^A+
% \begin{function}[rEXP,updated=2017-10-14]{style/fullwidth-stop}
%   \begin{fdusyntax}[emph={[1]fullwidth-stop}]
%     fullwidth-stop = (*<catcode|mapping|(false)>*)
%   \end{fdusyntax}
%^^A-
%   选择是否把全角实心句点\FSFW 作为默认的句号形状。
%   这种句号一般用于科技类文章,以避免与下标“$_o$”或“$_0$”混淆。
% \end{function}
%^^A!   Let full-width full stop ``\FSFW'' as the default full stop.
%^^A!   Generally, this punctuation is used for scientific articles,
%^^A!   where ``\FSID'' is easily to be confused with subscript
%^^A!   ``$_o$'' or ``$_0$''.
%^^A! \end{function}
%^^A!
%
% 选择 \kvopt{fullwidth-stop}{catcode} 或 \opt{mapping} 后,都会实现
% 上述效果。有所不同的是,在选择 \opt{catcode} 后,只有^^A
% \emph{显式的}\FSID 会被替换为\FSFW;但在选择 \opt{mapping} 后,
% \emph{所有的}\FSID 都会被替换。例如,如果您用宏保存了一些含有^^A
% \FSID 的文字,那么在选择 \opt{catcode} 时,其中的\FSID 不会被
% 替换为\FSFW。
%^^A! If you choose \kvopt{fullwidth-stop}{catcode}, only
%^^A! \emph{explicit} ``\FSID'' will be replaced by ``\FSFW''; when
%^^A! choosing \kvopt{fullwidth-stop}{mapping}, however, \emph{all}
%^^A! the ``\FSID'' will be replaced.
%^^A!
%
% 选项 \kvopt{fullwidth-stop}{mapping} 只在 \XeTeX{} 下有效。使用
% \LuaTeX{} 编译时,该选项相当于 \kvopt{fullwidth-stop}{catcode}。
%^^A! \opt{mapping} is valid only under \XeTeX{}. When compiling
%^^A! with \LuaTeX{}, it is equivalent to \opt{catcode}.
%^^A!
%
% 如果您在选择 \kvopt{fullwidth-stop}{mapping} 后仍需要临时显示^^A
% \FSID,可以按如下方法操作:
% \begin{latexexample}[moretexcs={\CJKfontspec},emph={[1]Mapping}]
%   % 请使用 XeTeX 编译
%   % 外侧的花括号表示分组
%   这是一个句号{\CJKfontspec{(*\meta{字体名}*)}[Mapping=full-stop]。}
% \end{latexexample}
%^^A! If you want to display ``\FSID'' temporarily after setting
%^^A! \kvopt{fullwidth-stop}{mapping}, the following code snippet
%^^A! will be helpful:
%^^A! \begin{latexexample}[moretexcs={\CJKfontspec},emph={[1]Mapping}]
%^^A!   % Compiled with XeTeX
%^^A!   % The outside braces is used for group
%^^A!   这是一个句号{\CJKfontspec{(*\meta{font name}*)}[Mapping=full-stop]。}
%^^A! \end{latexexample}
%^^A!
%
% \begin{function}{style/footnote-style}
%^^A 这里奇怪的东西是用来控制对齐的。fdusyntax 会吃掉开头的几个
%^^A 空格,因此这里用 X 来占位。
%   \begin{fdusyntax}[emph={[1]footnote-style}]
%     footnote-style = (*<plain|\\
%       XXXXXX\mbox{}~~~~~~~~~~~~~~~~~libertinus|libertinus*|libertinus-sans|\\
%       XXXXXX\mbox{}~~~~~~~~~~~~~~~~~pifont|pifont*|pifont-sans|pifont-sans*|\\
%       XXXXXX\mbox{}~~~~~~~~~~~~~~~~~xits|xits-sans|xits-sans*>*)
%   \end{fdusyntax}
%   设置脚注编号样式。西文字体设置会影响其默认取值(见
%   表~\ref{tab:footnote-font})。因此,要使得该选项生效,需将其
%   放置在 \opt{font} 选项之后。带有 |sans| 的为相应的无衬线字体
%   版本;带有 |*| 的为阴文样式(即黑底白字)。
% \end{function}
%^^A! \begin{function}{style/footnote-style}
%^^A!   \begin{fdusyntax}[emph={[1]footnote-style}]
%^^A!     footnote-style = (*<plain|\\
%^^A!       XXXX\mbox{}~~~~~~~~~~~~~~~~~libertinus|libertinus*|libertinus-sans|\\
%^^A!       XXXX\mbox{}~~~~~~~~~~~~~~~~~pifont|pifont*|pifont-sans|pifont-sans*|\\
%^^A!       XXXX\mbox{}~~~~~~~~~~~~~~~~~xits|xits-sans|xits-sans*>*)
%^^A!   \end{fdusyntax}
%^^A!   Set the style of footnote numbers. Note that western fonts
%^^A!   will affect its default value (see table~\ref{tab:footnote-font}),
%^^A!   so you may put it after |font| option. The one with |sans|
%^^A!   is for the corresponding sans-serif version, while |*|
%^^A!   for white on black version.
%^^A! \end{function}
%^^A!
%
% \begin{table}[ht]
%   \caption{西文字体与脚注编号样式默认值的对应关系}
%   \label{tab:footnote-font}
%   \centering
%   \begin{tabular}{ccccc}
%     \toprule
%     \textbf{西文字体设置} &
%       |libertinus| & |lm|     & |palatino| & |times| \\
%     \midrule
%     \textbf{脚注编号样式默认值} &
%       |libertinus| & |pifont| & |pifont|   & |xits|  \\
%     \bottomrule
%   \end{tabular}
% \end{table}
%^^A! \begin{table}[ht]
%^^A!   \caption{Relationship between option \opt{font} and the
%^^A!     default value of \opt{footnote-style}}
%^^A!   \label{tab:footnote-font}
%^^A!   \centering
%^^A!   \begin{tabular}{ccccc}
%^^A!     \toprule
%^^A!     \textbf{Western fonts settings} &
%^^A!       |libertinus| & |lm|     & |palatino| & |times| \\
%^^A!     \midrule
%^^A!     \textbf{Default value of footnote number style} &
%^^A!       |libertinus| & |pifont| & |pifont|   & |xits|  \\
%^^A!     \bottomrule
%^^A!   \end{tabular}
%^^A! \end{table}
%^^A!
%
%^^A+
% \begin{function}[added=2017-08-13]{style/hyperlink}
%   \begin{fdusyntax}[emph={[1]hyperlink}]
%     hyperlink = (*<border|(color)|none>*)
%   \end{fdusyntax}
%^^A-
%   设置超链接样式。\opt{border} 表示在超链接四周绘制方框;
%   \opt{color} 表示用彩色显示超链接;\opt{none} 表示没有特殊装饰,
%   可用于生成最终的打印版文稿。
% \end{function}
%^^A!   Set the style of hyperlinks. \opt{border} draws borders around
%^^A!   hyperlinks; \opt{color} displays hyperlinks in colorful text;
%^^A!   \opt{none} leads to plain text, which is useful when printing
%^^A!   the final document.
%^^A! \end{function}
%^^A!
%
% \begin{function}[added=2017-08-13,updated=2017-12-08]{style/hyperlink-color}
%   \begin{fdusyntax}[emph={[1]hyperlink-color}]
%     hyperlink-color = (*<(default)|classic|elegant|fantasy|material|\\
%       XXXXXX\mbox{}~~~~~~~~~~~~~~~~~~business|science|summer|autumn|graylevel|prl>*)
%   \end{fdusyntax}
%   设置超链接颜色。该选项在 \kvopt{hyperlink}{none} 时无效。
%   各选项所代表的颜色见表~\ref{tab:hyperlink-color}。
% \end{function}
%^^A! \begin{function}[added=2017-08-13,updated=2017-12-08]{style/hyperlink-color}
%^^A!   \begin{fdusyntax}[emph={[1]hyperlink-color}]
%^^A!     hyperlink-color = (*<(default)|classic|elegant|fantasy|material|\\
%^^A!       XXXX\mbox{}~~~~~~~~~~~~~~~~~~business|science|summer|autumn|graylevel|prl>*)
%^^A!   \end{fdusyntax}
%^^A!   Set the color of hyperlinks. It is invalid if
%^^A!   \kvopt{hyperlink}{none}. The related colors can be found
%^^A!   in table~\ref{tab:hyperlink-color}.
%^^A! \end{function}
%^^A!
%
%^^A+
% \begin{table}[ht]
% \centering
%^^A-
% \newcommand\linkcolorexam[3]{^^A
%   {\small 图~\textcolor[HTML]{#1}{1-2},
%     (\textcolor[HTML]{#1}{3.4})~式} &
%   {\small \textcolor[HTML]{#2}{\texttt{http://g.cn}}} &
%   {\small 文献~[\textcolor[HTML]{#3}{1}],
%     (\textcolor[HTML]{#3}{Knuth~1986})}}
%^^A! \newcommand\linkcolorexam[3]{%
%^^A!   {\small Fig.~\textcolor[HTML]{#1}{1-2},
%^^A!     Eq.~(\textcolor[HTML]{#1}{3.4})} &
%^^A!   {\small \textcolor[HTML]{#2}{\texttt{http://g.cn}}} &
%^^A!   {\small Ref.~[\textcolor[HTML]{#3}{1}],
%^^A!     (\textcolor[HTML]{#3}{Knuth~1986})}}
% \begin{threeparttable}
% \caption{预定义的超链接颜色方案}
% \label{tab:hyperlink-color}
%^^A! \begin{threeparttable}
%^^A! \caption{Pre-defined hyperlink color schemes}
%^^A! \label{tab:hyperlink-color}
% \begin{tabular}{c*{3}{>{\hspace{0.2cm}}c<{\hspace{0.2cm}}}}
%   \toprule
%   \textsf{选项} & \textsf{链接} & \textsf{URL} & \textsf{引用} \\
%^^A! \begin{tabular}{c*{3}{>{\hspace{0.2cm}}c<{\hspace{0.2cm}}}}
%^^A!   \toprule
%^^A!   \textsf{Options} & \textsf{Cross references} & \textsf{URL} & \textsf{Citation} \\
%^^A+
%   \midrule
%   \opt{default}            & \linkcolorexam{990000}{0000B2}{007F00} \\
%   \opt{classic}            & \linkcolorexam{FF0000}{0000FF}{00FF00} \\
%   \opt{elegant}\tnote{a}   & \linkcolorexam{961212}{C31818}{9B764F} \\
%   \opt{fantasy}\tnote{b}   & \linkcolorexam{FF4A19}{FF3F94}{934BA1} \\
%   \opt{material}\tnote{c}  & \linkcolorexam{E91E63}{009688}{4CAF50} \\
%   \opt{business}\tnote{d}  & \linkcolorexam{D14542}{295497}{1F6E43} \\
%   \opt{science}\tnote{e}   & \linkcolorexam{CA0619}{389F9D}{FF8920} \\
%   \opt{summer}\tnote{f}    & \linkcolorexam{00AFAF}{5F5FAF}{5F8700} \\
%   \opt{autumn}\tnote{f}    & \linkcolorexam{D70000}{D75F00}{AF8700} \\
%   \opt{graylevel}\tnote{c} & \linkcolorexam{616161}{616161}{616161} \\
%   \opt{prl}\tnote{g}       & \linkcolorexam{2D3092}{2D3092}{2D3092} \\
%   \bottomrule
% \end{tabular}
% \begin{tablenotes}
%^^A-
%   \item[a] 来自 \href{https://tex.stackexchange.com/}^^A
%     {\TeX{} - \LaTeX{} Stack Exchange 网站}。
%   \item[b] Adobe CC 产品配色。
%   \item[c] 取自 Material 色彩方案
%     (见 \url{https://material.io/guidelines/style/color.html})。
%   \item[d] Microsoft Office 2016 产品配色。
%   \item[e] 来自 \href{https://www.wolfram.com/}{Wolfram Research 网站}。
%   \item[f] 均取自 Solarized 色彩方案
%     (见 \url{http://ethanschoonover.com/solarized})。
%   \item[g] \textit{Physical Review Letter} 杂志配色。
%^^A!   \item[a] From \href{https://tex.stackexchange.com/}%
%^^A!     {\TeX{} - \LaTeX{} Stack Exchange}.
%^^A!   \item[b] Adobe CC.
%^^A!   \item[c] Material Design color palette
%^^A!     (See \url{https://material.io/guidelines/style/color.html}).
%^^A!   \item[d] Microsoft Office 2016.
%^^A!   \item[e] From \href{https://www.wolfram.com/}{Wolfram Research website}.
%^^A!   \item[f] Solarized color palette
%^^A!     (See \url{http://ethanschoonover.com/solarized}).
%^^A!   \item[g] \textit{Physical Review Letter} magazine.
%^^A+
% \end{tablenotes}
% \end{threeparttable}
% \end{table}
%^^A-
%^^A!
%
%^^A+
% \begin{function}[added=2018-01-25]{style/bib-backend}
%   \begin{fdusyntax}[emph={[1]bib-backend}]
%     bib-backend = (*<bibtex|biblatex>*)
%   \end{fdusyntax}
%^^A-
%   选择参考文献的支持方式。选择 \opt{bibtex} 后,将使用 \BibTeX{}
%   处理文献,样式由 \pkg{natbib} 宏包负责;选择 \opt{biblatex} 后,
%   将使用 \biber{} 处理文献,样式则由 \pkg{biblatex} 宏包负责。
% \end{function}
%^^A!   Specify the backend or driver of bibliography processing.
%^^A!   \BibTeX{} and \pkg{natbib} package will be used if you choose
%^^A!   \opt{bibtex}, while \biber{} and \pkg{biblatex} will be used
%^^A!   if you choose \opt{biblatex}.
%^^A! \end{function}
%^^A!
%
% \begin{function}[added=2017-10-28,updated=2018-01-25]^^A
%     {style/bib-style}
%   \begin{fdusyntax}[emph={[1]bib-style}]
%     bib-style = (*<author-year|(numerical)|\meta{其他样式}>*)
%   \end{fdusyntax}
%   设置参考文献样式。\opt{author-year} 和 \opt{numerical} 分别对应
%   国家标准 GB/T 7714--2015 \scite{gb-t-7714-2015} 中的著者—出版年制
%   和顺序编码制。选择 \meta{其他样式} 时,如果 \kvopt{bib-backend}^^A
%   {bibtex},需保证相应的 \file{.bst} 格式文件能被调用;而如果
%   \kvopt{bib-backend}{biblatex},则需保证相应的 \file{.bbx} 格式文件
%   能被调用。
% \end{function}
%^^A! \begin{function}[added=2017-10-28,updated=2018-01-25]%
%^^A!     {style/bib-style}
%^^A!   \begin{fdusyntax}[emph={[1]bib-style}]
%^^A!     bib-style = (*<author-year|(numerical)|\meta{other style}>*)
%^^A!   \end{fdusyntax}
%^^A!   Set the style of bibliography. \opt{author-year} and
%^^A!   \opt{numerical} will follow the standard GB/T 7714--2015.
%^^A!   By setting \kvopt{bib-style}{\meta{other style}}, you can use
%^^A!   other bibliography style (\file{.bst} file for
%^^A!   \kvopt{bib-backend}{bibtex} and \file{.bbx} file for
%^^A!   \kvopt{bib-backend}{biblatex}). Suffix is not needed.
%^^A! \end{function}
%^^A!
%
% \begin{function}[added=2018-01-25]{style/cite-style}
%   \begin{fdusyntax}[emph={[1]cite-style}]
%     cite-style = (*\marg{引用样式}*)
%   \end{fdusyntax}
%   选择引用格式。默认为空,即与参考文献样式(著者—出版年制或顺序
%   编码制)保持一致。如果手动填写,需保证相应的 \file{.cbx} 格式文件
%   能被调用。该选项在 \kvopt{bib-backend}{bibtex} 时无效。
% \end{function}
%^^A! \begin{function}[added=2018-01-25]{style/cite-style}
%^^A!   \begin{fdusyntax}[emph={[1]cite-style}]
%^^A!     cite-style = (*\marg{style}*)
%^^A!   \end{fdusyntax}
%^^A!   Select citation style. Default value is empty, which means
%^^A!   the citation style will follow your bibliography style
%^^A!   (author-year or numeric). If you want change the citation
%^^A!   style, the corresponding \file{.cbx} file must be available.
%^^A!   This option is invalid when \kvopt{bib-backend}{bibtex}.
%^^A! \end{function}
%^^A!
%
% \begin{function}[added=2018-01-25]{style/bib-resource}
%   \begin{fdusyntax}[emph={[1]bib-resource}]
%     bib-resource = (*\marg{文件}*)
%   \end{fdusyntax}
%   参考文献数据源。可以是单个文件,也可以是用英文逗号隔开的一组文件。
%   如果 \kvopt{bib-backend}{biblatex},则必须明确给出 \file{.bib}
%   后缀名。
% \end{function}
%^^A! \begin{function}[added=2018-01-25]{style/bib-resource}
%^^A!   \begin{fdusyntax}[emph={[1]bib-resource}]
%^^A!     bib-resource = (*\marg{bib file\symbol{"28}s\symbol{"29}}*)
%^^A!   \end{fdusyntax}
%^^A!   Specify the bibliography database (usually in \file{.bib}
%^^A!   format). If using more than one files, the file names should
%^^A!   be separated with comma. When \kvopt{bib-backend}{biblatex},
%^^A!   you must type in the ``\file{.bib}'' suffix.
%^^A! \end{function}
%^^A!
%
% \begin{function}[added=2017-08-10]{style/logo}
%   \begin{fdusyntax}[emph={[1]logo}]
%     logo = (*\marg{文件}*)
%   \end{fdusyntax}
%   封面中校名图片的文件名。默认值为 \file{fudan-name.pdf}。
% \end{function}
%^^A! \begin{function}[added=2017-08-10]{style/logo}
%^^A!   \begin{fdusyntax}[emph={[1]logo}]
%^^A!     logo = (*\marg{file}*)
%^^A!   \end{fdusyntax}
%^^A!   File name of the logo in thesis cover. Default value is
%^^A!   \file{fudan-name.pdf}.
%^^A! \end{function}
%^^A!
%
% \begin{function}[added=2017-08-10]{style/logo-size}
%   \begin{fdusyntax}[emph={[1]logo-size}]
%     logo-size = (*\marg{宽度}*)
%     logo-size = {(*\meta{宽度}*), (*\meta{高度}*)}
%   \end{fdusyntax}
%   校名图片的大小。默认仅指定了宽度,为 |0.5\textwidth|\/。
%   如果仅需指定高度,可在 \meta{宽度} 处填入一个空的分组 |{}|。
% \end{function}
%^^A! \begin{function}[added=2017-08-10]{style/logo-size}
%^^A!   \begin{fdusyntax}[emph={[1]logo-size}]
%^^A!     logo-size = (*\marg{width}*)
%^^A!     logo-size = {(*\meta{width}*), (*\meta{height}*)}
%^^A!   \end{fdusyntax}
%^^A!   Size of the logo. By default, only width is set to
%^^A!   |0.5\textwidth|. To set height only, you can put an
%^^A!   empty group ``|{}|'' at \meta{width}.
%^^A! \end{function}
%^^A!
%
%^^A+
% \begin{function}[added=2017-07-06]{style/auto-make-cover}
%   \begin{fdusyntax}[emph={[1]auto-make-cover}]
%     auto-make-cover = (*<\TTF>*)
%   \end{fdusyntax}
%^^A-
%   是否自动生成论文封面(封一)、指导小组成员名单(封二)和
%   声明页(封三)。封面中的各项信息,可通过 \cs{fdusetup} 录入,
%   具体请参阅 \ref{subsubsec:信息录入}~节。
% \end{function}
%^^A!   Whether generate thesis cover, list of instructors (inside
%^^A!   front cover) and declaration page (inside back cover)
%^^A!   automatically. Entries in the cover can be specified also
%^^A!   via \cs{fdusetup}, and you can find more details in
%^^A!   subsubsection~\ref{subsubsec:information}.
%^^A! \end{function}
%^^A!
%
% \begin{function}{\makecoveri,\makecoverii,\makecoveriii}
%   用于手动生成论文封面、指导小组成员名单和声明页。这几个命令不能
%   确保页码的正确编排,因此除非必要,您应当始终使用自动生成的封面。
% \end{function}
%^^A! \begin{function}{\makecoveri,\makecoverii,\makecoveriii}
%^^A!   For generating thesis cover, list of instructors and
%^^A!   declaration page manually. These commands cannot ensure
%^^A!   the correct page numbers, hence you should always use the
%^^A!   auto-generated thesis cover unless necessary.
%^^A! \end{function}
%^^A!
%
% \subsubsection{信息录入} \label{subsubsec:信息录入}
%^^A! \subsubsection{Personal information} \label{subsubsec:information}
%^^A!
%
% \begin{function}{info}
%   \begin{fdusyntax}[emph={[1]info}]
%     info = (*\marg{键值列表}*)
%     info/(*\meta{key}*) = (*\meta{value}*)
%   \end{fdusyntax}
%   该选项包含许多子项目,用于录入论文信息。具体内容见下。以下带“|*|”
%   的项目表示对应的英文字段。
% \end{function}
%^^A! \begin{function}{info}
%^^A!   \begin{fdusyntax}[emph={[1]info}]
%^^A!     info = (*\marg{key-value list}*)
%^^A!     info/(*\meta{key}*) = (*\meta{value}*)
%^^A!   \end{fdusyntax}
%^^A!   This general option is for entering your personal information.
%^^A!   See the following details. Note that options with ``|*|'' are
%^^A!   the corresponding English items.
%^^A! \end{function}
%^^A!
%
%^^A+
% \begin{function}[added=2018-02-01,updated=2019-03-12]{info/degree}
%   \begin{fdusyntax}[emph={[1]degree}]
%     degree = (*<(academic)|professional>*)
%   \end{fdusyntax}
%^^A-
%   学位类型,仅适用于博士和硕士学位论文。\opt{academic} 和 \opt{professional}
%   分别表示学术学位和专业学位。
% \end{function}
%^^A!   Degree type. This option can only be used in master degree
%^^A!   thesis.
%^^A! \end{function}
%^^A!
%
% \begin{function}{info/title,info/title*}
%   \begin{fdusyntax}[emph={[1]title,title*}]
%     title  = (*\marg{中文标题}*)
%     title* = (*\marg{英文标题}*)
%   \end{fdusyntax}
%   论文标题。默认会在约 20 个汉字字宽处强制断行,但为了语义的
%   连贯以及排版的美观,如果您的标题长于一行,建议使用“|\\|”
%   手动断行。
% \end{function}
%^^A! \begin{function}{info/title,info/title*}
%^^A!   \begin{fdusyntax}[emph={[1]title,title*}]
%^^A!     title  = (*\marg{title in Chinese}*)
%^^A!     title* = (*\marg{title in English}*)
%^^A!   \end{fdusyntax}
%^^A!   Title of your thesis. The line width is about \SI{30}{em} by
%^^A!   default, but you may break it with |\\| manually.
%^^A! \end{function}
%^^A!
%
% \begin{function}{info/author,info/author*}
%   \begin{fdusyntax}[emph={[1]author,author*}]
%     author  = (*\marg{姓名}*)
%     author* = (*\marg{英文姓名(或拼音)}*)
%   \end{fdusyntax}
%   作者姓名。
% \end{function}
%^^A! \begin{function}{info/author,info/author*}
%^^A!   \begin{fdusyntax}[emph={[1]author,author*}]
%^^A!     author  = (*\marg{name in Chinese}*)
%^^A!     author* = (*\marg{name in English \lparen or Pinyin\rparen}*)
%^^A!   \end{fdusyntax}
%^^A!   Author's name.
%^^A! \end{function}
%^^A!
%
% \begin{function}{info/supervisor}
%   \begin{fdusyntax}[emph={[1]supervisor}]
%     supervisor = (*\marg{姓名}*)
%   \end{fdusyntax}
%   导师姓名。
% \end{function}
%^^A! \begin{function}{info/supervisor}
%^^A!   \begin{fdusyntax}[emph={[1]supervisor}]
%^^A!     supervisor = (*\marg{name}*)
%^^A!   \end{fdusyntax}
%^^A!   Supervisor's name.
%^^A! \end{function}
%^^A!
%
% \begin{function}{info/department}
%   \begin{fdusyntax}[emph={[1]department}]
%     department = (*\marg{名称}*)
%   \end{fdusyntax}
%   院系名称。
% \end{function}
%^^A! \begin{function}{info/department}
%^^A!   \begin{fdusyntax}[emph={[1]department}]
%^^A!     department = (*\marg{name}*)
%^^A!   \end{fdusyntax}
%^^A!   Name of the department.
%^^A! \end{function}
%^^A!
%
% \begin{function}{info/major}
%   \begin{fdusyntax}[emph={[1]major}]
%     major = (*\marg{名称}*)
%   \end{fdusyntax}
%   专业名称。
% \end{function}
%^^A! \begin{function}{info/major}
%^^A!   \begin{fdusyntax}[emph={[1]major}]
%^^A!     major = (*\marg{name}*)
%^^A!   \end{fdusyntax}
%^^A!   Name of the major.
%^^A! \end{function}
%^^A!
%
% \begin{function}{info/student-id}
%   \begin{fdusyntax}[emph={[1]student-id}]
%     student-id = (*\marg{数字}*)
%   \end{fdusyntax}
%   作者学号。
% \end{function}
%^^A! \begin{function}{info/student-id}
%^^A!   \begin{fdusyntax}[emph={[1]student-id}]
%^^A!     student-id = (*\marg{number}*)
%^^A!   \end{fdusyntax}
%^^A!   Author's student ID.
%^^A! \end{function}
%^^A!
%
% 复旦大学学号共 11 位,前两位为入学年份,之后一位为学生类型
% 代码(博士生为 1,硕士生为 2,本科生为 3),接下来的五位为
% 专业代码,最后三位为顺序号。
%^^A! In Fudan University, student ID has 11 digits. The first two
%^^A! are the year of attendance; next one represents the student's
%^^A! type (1 for doctor, 2 for master and 3 for bachelor); the
%^^A! following five digits are major ID while the last three are
%^^A! serial number.
%^^A!
%
% \begin{function}{info/school-id}
%   \begin{fdusyntax}[emph={[1]school-id}]
%     school-id = (*\marg{数字}*)
%   \end{fdusyntax}
%   学校代码。默认值为 10246(这是复旦大学的学校代码)。
% \end{function}
%^^A! \begin{function}{info/school-id}
%^^A!   \begin{fdusyntax}[emph={[1]school-id}]
%^^A!     school-id = (*\marg{number}*)
%^^A!   \end{fdusyntax}
%^^A!   School ID. Default value is 10246 (school ID of Fudan University).
%^^A! \end{function}
%^^A!
%
% \begin{function}{info/date}
%   \begin{fdusyntax}[emph={[1]date}]
%     date = (*\marg{日期}*)
%   \end{fdusyntax}
%   论文完成日期。默认值为文档编译日期(\tn{today})。
% \end{function}
%^^A! \begin{function}{info/date}
%^^A!   \begin{fdusyntax}[emph={[1]date}]
%^^A!     date = (*\marg{date}*)
%^^A!   \end{fdusyntax}
%^^A!   Finish date of your thesis. Default value is the compilation
%^^A!   date (\tn{today}).
%^^A! \end{function}
%^^A!
%
%^^A+
% \begin{function}[added=2017-07-04]{info/secret-level}
%   \begin{fdusyntax}[emph={[1]secret-level}]
%     secret-level = (*<(none)|i|ii|iii>*)
%   \end{fdusyntax}
%^^A-
%   密级。\opt{i}、\opt{ii}、\opt{iii} 分别表示秘密、机密、绝密;
%   \opt{none} 表示论文不涉密,即不显示密级与保密年限。
% \end{function}
%^^A!   Secret level. \opt{i}, \opt{ii} and \opt{iii} means
%^^A!   ``秘密'' (secret), ``机密'' (confidential) and ``绝密''
%^^A!   (top secret) respectively. \opt{none} means your thesis is
%^^A!   not secret-related and secret level and year will not be
%^^A!   shown.
%^^A! \end{function}
%^^A!
%
% \begin{function}[added=2017-07-04]{info/secret-year}
%   \begin{fdusyntax}[emph={[1]secret-year}]
%     secret-year = (*\marg{年限}*)
%   \end{fdusyntax}
%   保密年限。建议您使用中文,如“五年”。该选项在设置
%   \kvopt{secret-level}{none} 时无效。
% \end{function}
%^^A! \begin{function}[added=2017-07-04]{info/secret-year}
%^^A!   \begin{fdusyntax}[emph={[1]secret-year}]
%^^A!     secret-year = (*\marg{year}*)
%^^A!   \end{fdusyntax}
%^^A!   Secret year. It's recommended to use Chinese word as ``五年''
%^^A!   (5 years) here. This option is invalid if you have set
%^^A!   \kvopt{secret-level}{none}.
%^^A! \end{function}
%^^A!
%
% \begin{function}{info/instructors}
%   \begin{fdusyntax}[emph={[1]instructors}]
%     instructors = (*\marg{成员 1, 成员 2, ...}*)
%   \end{fdusyntax}
%   指导小组成员。各成员之间需使用英文逗号隔开。为防止歧义,
%   可以用分组括号“|{...}|”把各成员字段括起来。
% \end{function}
%^^A! \begin{function}{info/instructors}
%^^A!   \begin{fdusyntax}[emph={[1]instructors}]
%^^A!     instructors = (*\marg{member 1, member 2, ...}*)
%^^A!   \end{fdusyntax}
%^^A!   Instructors' name. Each name should be separated with
%^^A!   comma. To disambiguate, you may put text containing comma
%^^A!   into a group ``|{...}|''.
%^^A! \end{function}
%^^A!
%
% \begin{function}{info/keywords,info/keywords*}
%   \begin{fdusyntax}[emph={[1]keywords,keywords*}]
%     keywords  = (*\marg{中文关键字}*)
%     keywords* = (*\marg{英文关键字}*)
%   \end{fdusyntax}
%   关键字列表。各关键字之间需使用英文逗号隔开。为防止歧义,
%   可以用分组括号“|{...}|”把各字段括起来。
% \end{function}
%^^A! \begin{function}{info/keywords,info/keywords*}
%^^A!   \begin{fdusyntax}[emph={[1]keywords,keywords*}]
%^^A!     keywords  = (*\marg{keywords in Chinese}*)
%^^A!     keywords* = (*\marg{keywords in English}*)
%^^A!   \end{fdusyntax}
%^^A!   Keywords list. Each keyword should be separated with comma.
%^^A!   To disambiguate, you may put text containing comma into a
%^^A!   group ``|{...}|''.
%^^A! \end{function}
%^^A!
%
% \begin{function}{info/clc}
%   \begin{fdusyntax}[emph={[1]clc}]
%     clc = (*\marg{分类号}*)
%   \end{fdusyntax}
%   中图分类号(CLC)。
% \end{function}
%^^A! \begin{function}{info/clc}
%^^A!   \begin{fdusyntax}[emph={[1]clc}]
%^^A!     clc = (*\marg{classification codes}*)
%^^A!   \end{fdusyntax}
%^^A!   Chinese Library Classification (CLC).
%^^A! \end{function}
%^^A!
%
% \subsection{正文编写}
%^^A! \subsection{Writing your thesis}
%^^A!
%
% \begin{quotation}
%   喬孟符(吉)博學多能,以樂府稱。嘗云:「作樂府亦有法,曰^^A
%   \CJKunderdot{鳳頭、豬肚、豹尾}六字是也。」大概起要美麗,中要浩蕩,
%   結要響亮。尤貴在首尾貫穿,意思清新。苟能若是,斯可以言樂府矣。
% \end{quotation}
% \hfill ——陶宗儀《南村輟耕錄·作今樂府法》
%
% \subsubsection{凤头}
%^^A! \subsubsection{Front matter}
%^^A!
%
% \begin{function}{\frontmatter}
%   声明前置部分开始。
% \end{function}
%^^A! \begin{function}{\frontmatter}
%^^A!   Declare the beginning of front matter.
%^^A! \end{function}
%^^A!
%
% 在本模板中,前置部分包含目录、中英文摘要以及符号表等。
% 前置部分的页码采用小写罗马字母,并且与正文分开计数。
%^^A! Front matter contains table of contents, abstracts and notation
%^^A! list. The page numbers in front matter will be shown in
%^^A! lowercase Roman numerals, and will be counted separately with
%^^A! main matter.
%^^A!
%
% \begin{function}{\tableofcontents}
%   生成目录。为了生成完整、正确的目录,您至少需要编译\emph{两次}。
% \end{function}
%^^A! \begin{function}{\tableofcontents}
%^^A!   Generate the table of contents (TOC). You need to compile
%^^A!   the source file at least \emph{twice} to get the correct TOC.
%^^A! \end{function}
%^^A!
%
%^^A TODO: \DescribeEnv{abstract}
%^^A TODO: \DescribeEnv{abstract*}
% \begin{function}{abstract}
%   \begin{fdusyntax}[emph={[2]abstract}]
%     % 中文论文模板 (fduthesis)      % 英文论文模板 (fduthesis-en)
%     \begin{abstract}                \begin{abstract}
%       (*\meta{中文摘要} \hspace{3.52cm} \meta{英文摘要}*)
%     \end{abstract}                  \end{abstract}
%   \end{fdusyntax}
% \end{function}
% \begin{function}[rEXP]{abstract*}
%   \begin{fdusyntax}[emph={[2]abstract*}]
%     % 中文论文模板 (fduthesis)
%     \begin{abstract*}
%       (*\meta{英文摘要}*)
%     \end{abstract*}
%   \end{fdusyntax}
%   摘要。中文模板中,不带星号和带星号的版本分别用来输入中文摘要
%   和英文摘要;英文模板中没有带星号的版本,您只需输入英文摘要。
% \end{function}
%^^A! \begin{function}{abstract}
%^^A!   \begin{fdusyntax}[emph={[2]abstract}]
%^^A!     % fduthesis (Chinese thesis)    % fduthesis-en (English thesis)
%^^A!     \begin{abstract}                \begin{abstract}
%^^A!       (*\meta{Chinese abstract} \hspace{3cm} \meta{English abstract}*)
%^^A!     \end{abstract}                  \end{abstract}
%^^A!   \end{fdusyntax}
%^^A! \end{function}
%^^A! \begin{function}[rEXP]{abstract*}
%^^A!   \begin{fdusyntax}[emph={[2]abstract*}]
%^^A!     % Only for fduthesis
%^^A!     \begin{abstract*}
%^^A!       (*\meta{English abstract}*)
%^^A!     \end{abstract*}
%^^A!   \end{fdusyntax}
%^^A!   Abstract environment. In \cls{fduthesis}, \env{abstract} and
%^^A!   \env{abstract*} are used for Chinese and English abstract,
%^^A!   respectively; while in \cls{fduthesis-en}, there is no
%^^A!   \env{abstract*} environment and you need to write the English
%^^A!   abstract merely.
%^^A! \end{function}
%^^A!
%
% 摘要的最后,会显示关键字列表以及中图分类号(CLC)。
% 这两项可通过 \cs{fdusetup} 录入,具体
% 请参阅 \ref{subsubsec:信息录入}~节。
%^^A! At the end of abstract (both Chinese and English, if available),
%^^A! keywords list and CLC number will be shown. They can be
%^^A! specified via command \cs{fdusetup} and you may refer to
%^^A! subsubsection~\ref{subsubsec:information} for more details.
%^^A!
%
%^^A TODO: \DescribeEnv{notation}
% \begin{function}{notation}
%   \begin{fdusyntax}[emph={[2]notation}]
%     \begin{notation}(*\oarg{列格式说明}*)
%       (*\meta{符号 1}*)  &  (*\meta{说明}*)  \\
%       (*\meta{符号 2}*)  &  (*\meta{说明}*)  \\
%       (*\phantom{\meta{符号 $n$}}*)  (*$\vdots$*)
%       (*\meta{符号\ \kern-0.1em$n$}*)  &  (*\meta{说明}*)
%     \end{notation}
%   \end{fdusyntax}
%   符号表。可选参数 \meta{列格式说明}与 \LaTeX{} 中标准表格的列格
%   式说明语法一致,默认值为“|lp{7.5cm}|”,即第一列宽度自动调整,
%   第二列限宽 \SI{7.5}{cm},两列均为左对齐。
% \end{function}
%^^A! \begin{function}{notation}
%^^A!   \begin{fdusyntax}[emph={[2]notation}]
%^^A!     \begin{notation}(*\oarg{column format}*)
%^^A!       (*\meta{symbol 1}*)  &  (*\meta{description}*)  \\
%^^A!       (*\meta{symbol 2}*)  &  (*\meta{description}*)  \\
%^^A!       (*\phantom{\meta{symbol $n$}}*)  (*$\vdots$*)
%^^A!       (*\meta{symbol \kern-0.1em$n$}*)  &  (*\meta{description}*)
%^^A!     \end{notation}
%^^A!   \end{fdusyntax}
%^^A!   Notation list (or symbol list, nomenclature) environment.
%^^A!   The optional argument \meta{column format} is the same as
%^^A!   in a standard \LaTeX{} table. The default value is
%^^A!   ``|lp{7.5cm}|'', which means auto-width for the first column
%^^A!   and fix-width (\SI{7.5}{cm}) for the second; both columns will
%^^A!   be left-aligned.
%^^A! \end{function}
%^^A!
%
% \subsubsection{猪肚}
%^^A! \subsubsection{Main matter}
%^^A!
%
% \begin{function}{\mainmatter}
%   声明主体部分开始。
% \end{function}
%^^A! \begin{function}{\mainmatter}
%^^A!   Declare the beginning of main matter.
%^^A! \end{function}
%^^A!
%
% 主体部分是论文的核心,您可以分章节撰写。如有需求,也可以采用
% 多文件编译的方式。主体部分的页码采用阿拉伯数字。
%^^A! As the name suggests, ``main matter'' is the main body of your
%^^A! thesis. When working on a big projects, it's usually a good
%^^A! idea to split the source file into several parts. The page
%^^A! numbers in main matter are shown in arabic numerals.
%^^A!
%
% \begin{function}[updated=2018-01-15]{\footnote}
%   \begin{fdusyntax}[deletetexcs={\footnote},%
%       morekeywords={\footnote}]
%     \footnote(*\marg{脚注文字}*)
%   \end{fdusyntax}
%   插入脚注。脚注编号样式可利用 \opt{style/footnote-style} 选项控制,
%   具体见 \ref{subsubsec:论文格式}~小节。
% \end{function}
%^^A! \begin{function}[updated=2018-01-15]{\footnote}
%^^A!   \begin{fdusyntax}[deletetexcs={\footnote},%
%^^A!       morekeywords={\footnote}]
%^^A!     \footnote(*\marg{text}*)
%^^A!   \end{fdusyntax}
%^^A!   Insert a footnote. The style of footnote numbers can be set
%^^A!   with option \opt{style/foot\-note\-style}. See subsubsection~%
%^^A!   \ref{subsubsec:style-and-format} for more details.
%^^A! \end{function}
%^^A!
%
%^^A TODO: \DescribeEnv{proof}
% \begin{function}{axiom,corollary,definition,example,lemma,
%   proof,theorem}
%   \begin{fdusyntax}[emph={[2]proof}]
%     \begin{proof}(*\oarg{小标题}*)
%       (*\meta{证明过程}*)
%     \end{proof}
%   \end{fdusyntax}
%   一系列预定义的数学环境。具体含义见表~\ref{tab:theorem}。
% \end{function}
%^^A! \begin{function}{axiom,corollary,definition,example,lemma,
%^^A!   proof,theorem}
%^^A!   \begin{fdusyntax}[emph={[2]proof}]
%^^A!     \begin{proof}(*\oarg{subheading}*)
%^^A!       (*\meta{procedure of proof}*)
%^^A!     \end{proof}
%^^A!   \end{fdusyntax}
%^^A!   A series of pre-defined math environments.
%^^A! \end{function}
%^^A!
%
% \begin{table}[ht]
%   \caption{预定义的数学环境} \label{tab:theorem}
%   \centering
%   \begin{tabular}{cccccccc}
%     \toprule
%     \textbf{名称} &
%       \env{axiom}   & \env{corollary} & \env{definition} &
%       \env{example} & \env{lemma}     & \env{proof}      &
%       \env{theorem} \\
%     \midrule
%     \textbf{含义} &
%       公理 & 推论 & 定义 & 例 & 引理 & 证明 & 定理 \\
%     \bottomrule
%   \end{tabular}
% \end{table}
%
% 证明环境(\env{proof})的最后会添加证毕符号“$\QED$”。要确保
% 该符号在正确的位置显示,您需要按照 \ref{subsec:编译方式}~节
% 中的有关说明编译\emph{两次}。
%^^A! A QED\footnote{Abbreviation of Latin phrase \emph{quod erat
%^^A!   demonstrandum}, means ``what was to be demonstrated''.}
%^^A! symbol ``$\QED$'' will be added at the end of \env{proof}
%^^A! environment. You need to compile the source file \emph{twice}
%^^A! as in subsection~\ref{subsec:compilation} in order to make
%^^A! the position of QED symbol correct.
%^^A!
%
% \begin{function}[updated=2017-12-12]{\newtheorem}
%   \begin{fdusyntax}[deletetexcs={\newtheorem},
%       morekeywords={\newtheorem,\newtheorem*}]
%     \newtheorem(*\oarg{选项}\marg{环境名}\marg{标题}*)
%     \newtheorem*(*\oarg{选项}\marg{环境名}\marg{标题}*)
%     \begin(*\marg{环境名}\oarg{小标题}*)
%       (*\meta{内容}*)
%     \end(*\marg{环境名}*)
%   \end{fdusyntax}
%   声明新的定理类环境(数学环境)。带星号的版本表示不进行编号,
%   并且会默认添加证毕符号“$\QED$”。声明后,即可同预定义的数学环境
%   一样使用。
% \end{function}
%^^A! \begin{function}[updated=2017-12-12]{\newtheorem}
%^^A!   \begin{fdusyntax}[deletetexcs={\newtheorem},
%^^A!       morekeywords={\newtheorem,\newtheorem*}]
%^^A!     \newtheorem(*\oarg{options}\marg{environment}\marg{title}*)
%^^A!     \newtheorem*(*\oarg{options}\marg{environment}\marg{title}*)
%^^A!     \begin(*\marg{environment}\oarg{subheading}*)
%^^A!       (*\meta{contents}*)
%^^A!     \end(*\marg{environment}*)
%^^A!   \end{fdusyntax}
%^^A!   Declare new math environments (theorems). If you use
%^^A!   \cs{newtheorem*}, then the theorem will not be numbered, and
%^^A!   a QED symbol ``$\QED$'' will be added at the end of the
%^^A!   environment. All the theorem environments defined by yourself
%^^A!   can be used as the pre-defined ones.
%^^A! \end{function}
%^^A!
%
% 事实上,表~\ref{tab:theorem} 中预定义的环境正是通过以下方式定义的:
% \begin{latexexample}[deletetexcs={\newtheorem},
%     morekeywords={\newtheorem,\newtheorem*}]
%   \newtheorem*{proof}{证明}
%   \newtheorem{axiom}{公理}
%   \newtheorem{corollary}{定理}
%   ...
% \end{latexexample}
%^^A! Actually, the pre-defined math environments are just defined
%^^A! with \cs{new\-the\-o\-rem} and \cs{new\-the\-o\-rem*}:
%^^A! \begin{latexexample}[deletetexcs={\newtheorem},
%^^A!     morekeywords={\newtheorem,\newtheorem*}]
%^^A!   \newtheorem*{proof}{proof}
%^^A!   \newtheorem{axiom}{axiom}
%^^A!   \newtheorem{corollary}{corollary}
%^^A!   ...
%^^A! \end{latexexample}
%^^A!
%
% 与 \cs{fdusetup} 相同,\cs{newtheorem} 的可选参数 \meta{选项}
% 也为一组键值列表。可用的选项见下。注意您无需输入“|theorem/|”。
%^^A! Similar with \cs{fdusetup}, the optional argument \meta{options}
%^^A! of \cs{newtheorem} is a key-value list as well. The available
%^^A! are described below. Note that you don't need to type in the
%^^A! ``|theorem/|'' prefix.
%^^A!
%
% \begin{function}{theorem/style}
%   \begin{fdusyntax}[emph={[1]style}]
%     style = (*<(plain)|margin|change|\\
%       XXXXXX\mbox{}~~~~~~~~break|marginbreak|changebreak>*)
%   \end{fdusyntax}
%   定理类环境的总体样式。
% \end{function}
%^^A! \begin{function}{theorem/style}
%^^A!   \begin{fdusyntax}[emph={[1]style}]
%^^A!     style = (*<(plain)|margin|change|\\
%^^A!       XXXX\mbox{}~~~~~~~~break|marginbreak|changebreak>*)
%^^A!   \end{fdusyntax}
%^^A!   The overall style of the theorem environment.
%^^A! \end{function}
%^^A!
%
% \begin{function}{theorem/header-font}
%   \begin{fdusyntax}[emph={[1]header-font}]
%     header-font = (*\marg{字体}*)
%   \end{fdusyntax}
%   定理头(即标题)的字体。中文模板默认为 \tn{sffamily},即无衬线体
%   (黑体);英文模板默认为 |\bfseries\upshape|,即加粗直立体。
% \end{function}
%^^A! \begin{function}{theorem/header-font}
%^^A!   \begin{fdusyntax}[emph={[1]header-font}]
%^^A!     header-font = (*\marg{font}*)
%^^A!   \end{fdusyntax}
%^^A!   Font of the theorem header. Default value is \tn{sffamily}
%^^A!   and |\bfseries\upshape| for Chinese and English template,
%^^A!   respectively.
%^^A! \end{function}
%^^A!
%
% \begin{function}{theorem/body-font}
%   \begin{fdusyntax}[emph={[1]body-font}]
%     body-font = (*\marg{字体}*)
%   \end{fdusyntax}
%   定理内容的字体。中文模板默认为 \tn{fdu@kai},即楷体;英文模板
%   默认为 \tn{itshape},即斜体。
% \end{function}
%^^A! \begin{function}{theorem/body-font}
%^^A!   \begin{fdusyntax}[emph={[1]body-font}]
%^^A!     body-font = (*\marg{font}*)
%^^A!   \end{fdusyntax}
%^^A!   Font of the theorem body. Default value is \tn{fdu@kai}
%^^A!   (\textit{楷体}) and \tn{itshape} for Chinese and English
%^^A!   template, respectively.
%^^A! \end{function}
%^^A!
%
% \begin{function}{theorem/qed}
%   \begin{fdusyntax}[emph={[1]qed}]
%     qed = (*\marg{符号}*)
%   \end{fdusyntax}
%   定理结束标记(即证毕符号)。如果用 \cs{newtheorem} 声明定理,
%   则默认为空;用 \cs{newtheorem*} 声明,则默认为
%   |\ensuremath{\QED}|,即“$\QED$”。
% \end{function}
%^^A! \begin{function}{theorem/qed}
%^^A!   \begin{fdusyntax}[emph={[1]qed}]
%^^A!     qed = (*\marg{symbol}*)
%^^A!   \end{fdusyntax}
%^^A!   Theorem end mark. For \cs{newtheorem}, default value is
%^^A!   empty; for \cs{newtheorem*}, default value is
%^^A!   |\ensuremath{\QED}| (i.e.\ ``$\QED$'').
%^^A! \end{function}
%^^A!
%
% \begin{function}{theorem/counter}
%   \begin{fdusyntax}[emph={[1]counter}]
%     counter = (*\marg{计数器}*)
%   \end{fdusyntax}
%   定理计数器,表示定理编号在 \meta{计数器} 的下一级,并会随
%   \meta{计数器} 的变化而清零。\scite{刘海洋2013latex入门}
%   默认为 |chapter|,表示按章编号。使用 \cs{newtheorem*} 时,
%   该选项无效。
% \end{function}
%^^A! \begin{function}{theorem/counter}
%^^A!   \begin{fdusyntax}[emph={[1]counter}]
%^^A!     counter = (*\marg{counter}*)
%^^A!   \end{fdusyntax}
%^^A!   The theorem will be enumerated within \meta{counter}. For
%^^A!   example, the default value is |chapter|, which means with
%^^A!   each new \tn{chapter}, the enumeration begins again with 1.
%^^A!   This option is invalid for \cs{newtheorem*}.
%^^A! \end{function}
%^^A!
%
% \begin{function}{\caption}
%   \begin{fdusyntax}[deletetexcs={\caption},morekeywords={\caption}]
%     \caption(*\marg{图表标题}*)
%     \caption(*\oarg{短标题}\marg{长标题}*)
%   \end{fdusyntax}
%   插入图表标题。可选参数 \meta{短标题} 用于图表目录。在
%   \meta{长标题} 中,您可以进行长达多段的叙述;但 \meta{短标题}
%   和单独的 \meta{图表标题} 中则不允许分段。
%   \scite{刘海洋2013latex入门}
% \end{function}
%^^A! \begin{function}{\caption}
%^^A!   \begin{fdusyntax}[deletetexcs={\caption},morekeywords={\caption}]
%^^A!     \caption(*\marg{caption}*)
%^^A!     \caption(*\oarg{short caption}\marg{long caption}*)
%^^A!   \end{fdusyntax}
%^^A!   Insert the caption of figure or table. The optional argument
%^^A!   \meta{short caption} will be shown in the list of figures/tables.
%^^A!   In \meta{long caption}, you can write descriptions for several
%^^A!   paragraphs, but \meta{short caption} and the single
%^^A!   \meta{caption} will not allow multi-paragraph text (i.e.\
%^^A!   text containing \tn{par}) inside.
%^^A! \end{function}
%^^A!
%
% 按照排版惯例,建议您将表格的标题放置在绘制表格的命令之前,
% 而将图片的标题放置在绘图或插图的命令之后。另需注意,
% \tn{caption} 命令必须放置在浮动体环境(如 \env{table} 和
% \env{figure})中。
%^^A! By convention, caption of a table is usually put \emph{before}
%^^A! the table itself, while for figure it's the opposite.
%^^A! In addition, command \tn{caption} must be put inside float
%^^A! environments (e.g.\ \env{table} and \env{figure}).
%^^A!
%
% \subsubsection{豹尾}
%^^A! \subsubsection{Back matter}
%^^A!
%
% \begin{function}{\backmatter}
%   声明后置部分开始。
% \end{function}
%^^A! \begin{function}{\backmatter}
%^^A!   Declare the beginning of back matter.
%^^A! \end{function}
%^^A!
%
% 后置部分包含参考文献、声明页等。
%^^A! Back matter contains bibliography, declaration page, etc.
%^^A!
%
% \begin{function}[updated=2018-01-25]{\printbibliography}
%   \begin{fdusyntax}[morekeywords={\printbibliography}]
%     \printbibliography(*\oarg{选项}*)
%   \end{fdusyntax}
%   打印参考文献列表。如果 \kvopt{bib-backend}{bibtex},则 \meta{选项}
%   无效,相当于 \tn{bibliography} \texttt{\marg{文献数据库}},其中的
%   \meta{文献数据库} 可利用 \opt{style/bib-resource} 选项指定,具体见
%   \ref{subsubsec:论文格式}~小节;而如果 \kvopt{bib-backend}^^A
%   {biblatex},则该命令由 \pkg{biblatex} 宏包直接提供,可用选项请
%   参阅其文档 \cite{biblatex}。
% \end{function}
%^^A! \begin{function}[updated=2018-01-25]{\printbibliography}
%^^A!   \begin{fdusyntax}[morekeywords={\printbibliography}]
%^^A!     \printbibliography(*\oarg{options}*)
%^^A!   \end{fdusyntax}
%^^A!   Print the bibliography. When \kvopt{bib-backend}{bibtex}, then
%^^A!   \meta{options} is invalid and this command is equivalent to
%^^A!   \tn{bibliography} \texttt{\marg{bib files}}, where
%^^A!   \meta{bib files} should be specified with option
%^^A!   \opt{style/bib-resource} (see subsubsection~%
%^^A!   \ref{subsubsec:style-and-format}). When \kvopt{bib-backend}%
%^^A!   {bibtex}, then \tn{printbibliography} is provided by
%^^A!   \pkg{biblatex} and the available options can be found in its
%^^A!   documentation.
%^^A! \end{function}
%^^A!
%
% \section{宏包依赖情况}
%^^A! \section{Packages dependencies}
%^^A!
%
% 使用不同编译方式、指定不同选项,会导致宏包依赖情况有所不同。
% 具体如下:
% \begin{itemize}
%   \item 在任何情况下,本模板都会\emph{显式}调用以下宏包
%     (或文档类):
%     \begin{itemize}
%       \item \pkg{expl3}、\pkg{xparse}、\pkg{xtemplate} 和
%         \pkg{l3keys2e},用于构建 \LaTeX3 编程环境
%         \scite{source3}。它们分属 \pkg{l3kernel} 和
%         \pkg{l3packages} 宏集。
%       \item \cls{ctexbook},提供中文排版的通用框架。属于 \CTeX{}
%         宏集 \scite{CTeX}。
%       \item \pkg{amsmath},对 \LaTeX{} 的数学排版功能进行了
%         全面扩展。属于 \AmSLaTeX{} 套件。
%       \item \pkg{unicode-math},负责处理 Unicode 编码的
%         OpenType 数学字体。
%       \item \pkg{geometry},用于调整页面尺寸。
%       \item \pkg{fancyhdr},处理页眉页脚。
%       \item \pkg{footmisc},处理脚注。
%       \item \pkg{ntheorem},提供增强版的定理类环境。
%       \item \pkg{graphicx},提供图形插入的接口。
%       \item \pkg{longtable},长表格(允许跨页)支持。
%       \item \pkg{caption},用于设置题注。
%       \item \pkg{xcolor},提供彩色支持。
%       \item \pkg{hyperref},提供交叉引用、超链接、电子书签等功能。
%     \end{itemize}
%   \item 开启 \kvopt{style/footnote-style}{pifont} 后,会调用
%     \pkg{pifont} 宏包。它属于 \pkg{psnfss} 套件。
%   \item 开启 \kvopt{style/bib-backend}{bibtex} 后,会调用
%     \pkg{natbib} 宏包,并依赖 \BibTeX{} 程序。参考文献样式由
%     \pkg{gbt7714} 宏包提供 \scite{natbib,gbt7714}。
%   \item 开启 \kvopt{style/bib-backend}{biblatex} 后,会调用
%     \pkg{biblatex} 宏包,并依赖 \biber{} 程序。参考文献样式由
%     \pkg{biblatex-gb7714-2015} 宏包提供
%     \scite{biblatex,biblatex-gb7714-2015}。
% \end{itemize}
%^^A! Different compilation methods and options will result in a
%^^A! different packages dependency. Details are as follows:
%^^A! \begin{itemize}
%^^A!   \item In any case, \cls{fduthesis} will load the following
%^^A!     packages \emph{explicitly}:
%^^A!     \begin{itemize}
%^^A!       \item \pkg{expl3}, \pkg{xparse}, \pkg{xtemplate} and
%^^A!         \pkg{l3keys2e}, belong to \pkg{l3kernel} and
%^^A!         \pkg{l3packages} bundles
%^^A!       \item \cls{ctexbook}, belongs to \CTeX{} bundle
%^^A!       \item \pkg{amsmath}, belongs to \AmSLaTeX{} bundle
%^^A!       \item \pkg{unicode-math}
%^^A!       \item \pkg{geometry}
%^^A!       \item \pkg{fancyhdr}
%^^A!       \item \pkg{footmisc}
%^^A!       \item \pkg{ntheorem}
%^^A!       \item \pkg{graphicx}
%^^A!       \item \pkg{longtable}
%^^A!       \item \pkg{caption}
%^^A!       \item \pkg{xcolor}
%^^A!       \item \pkg{hyperref}
%^^A!     \end{itemize}
%^^A!   \item When chosen \kvopt{style/footnote-style}{pifont},
%^^A!     package \pkg{pifont} will be loaded. It belongs to
%^^A!     \pkg{psnfss} bundle.
%^^A!   \item When chosen \kvopt{style/bib-backend}{bibtex},
%^^A!     package \pkg{natbib} will be loaded. Meanwhile, program
%^^A!     \BibTeX{} will be required for compilation. The
%^^A!     bibliography style is provided by \pkg{gbt7714}.
%^^A!   \item When chosen \kvopt{style/bib-backend}{biblatex},
%^^A!     package \pkg{biblatex} will be loaded. Program \biber{}
%^^A!     will be required then. The bibliography style is provided
%^^A!     by \pkg{biblatex-gb7714-2015}.
%^^A! \end{itemize}
%^^A!
%
% 这里只列出了本模板直接调用的宏包。这些宏包自身的调用情况,
% 此处不再具体展开。如有需要,请参阅相关文档。
%^^A! Only the packages loaded directly by \cls{fduthesis} are listed
%^^A! here. If you need to know the dependencies of the packages
%^^A! themselves, please refer to the corresponding manuals.
%^^A!
%
% \begin{thebibliography}{99}
%
% \newcommand\urlprefix{\newline\hspace*{\fill}}
% \let\OldUrl=\url
% \renewcommand\url[2][]{{\small\textit{#1}~\OldUrl{#2}}}
% \newcommand\CTANurl[2][]{{\small\textit{#1}~\href{http://mirror.ctan.org/#2}^^A
%   {\ttfamily CTAN://#2}}}
%
% \subsection{图书}
%
% \bibitem{knuth1986texbook}
% \textsc{Knuth D E}.
% \newblock \textit{The \TeX book: Computers \& Typesetting, volume A} [M].
% \newblock Boston: Addison--Wesley Publishing Company, 1986
% \urlprefix \CTANurl[源代码^^A
%   \footnote{此代码只可作为学习之用。未经 Knuth 本人同意,您不应当编译此文档。}:]^^A
%   {systems/knuth/dist/tex/texbook.tex}
%
% \bibitem{mittelbach2004latexcompanion}
% \textsc{Mittelbach F} and \textsc{Goossens M}.
% \newblock \textit{The \LaTeX{} Companion} [M].
% \newblock 2nd ed.
% \newblock Boston: Addison--Wesley Publishing Company, 2004
%
% \bibitem{胡伟2017latex2e}
% 胡伟.
% \newblock \textit{\LaTeXe{} 文类和宏包学习手册} [M].
% \newblock 北京: 清华大学出版社, 2017
%
% \bibitem{刘海洋2013latex入门}
% 刘海洋.
% \newblock \textit{\LaTeX{} 入门} [M].
% \newblock 北京: 电子工业出版社, 2013
%
% \subsection{标准、规范}
%
% \bibitem{gb-t-7713.1-2006}
% 国务院学位委员会办公室, 全国信息与文献标准化技术委员会.
% \newblock \textit{学位论文编写规则: GB/T 7713.1--2006} [S].
% \newblock 北京: 中国标准出版社, 2007
%
% \bibitem{gb-t-7714-2015}
% 全国信息与文献标准化技术委员会.
% \newblock \textit{信息与文献\quad 参考文献著录规则: GB/T 7714--2015} [S].
% \newblock 北京: 中国标准出版社, 2015
%
% \bibitem{gb-t-15834-2011}
% 教育部语言文字信息管理司.
% \newblock \textit{标点符号用法: GB/T 15834--2011} [S/OL].
% \newblock 北京: 中国标准出版社, 2012
% \urlprefix\url{http://www.moe.gov.cn/ewebeditor/uploadfile/2015/01/13/20150113091548267.pdf}
%
% \bibitem{clreq}
% W3C.
% \newblock \textit{中文排版需求(Requirements for Chinese Text Layout)} [EB/OL].
% \newblock (2019-03-13) 
% \urlprefix\url{https://w3c.github.io/clreq/}
%
% \bibitem{复旦大学论文规范}
% 复旦大学图书馆, 复旦大学研究生院.
% \newblock \textit{复旦大学博士、硕士学位论文规范} [EB/OL].
% \newblock 2017 年 3 月修订版.
% \newblock (2017-03-27) 
% \urlprefix\url{http://www.gs.fudan.edu.cn/_upload/article/4c/a8/a82545ef443b9c057c14ba13782c/c883c6f3-6d7f-410c-8f30-d8bde6fcb990.doc}
%
% \subsection{宏包、模版}
%
% \bibitem{source2e}
% \textsc{Braams J}, \textsc{Carlisle D}, \textsc{Jeffrey A}, et al.
% \newblock \textit{The \LaTeXe{} Sources} [CP/OL].
% \newblock (2018-12-01) 
% \urlprefix\url{https://ctan.org/pkg/latex}
% \urlprefix\CTANurl[源代码:]{macros/latex/base/source2e.pdf}
%
% \bibitem{CTeX}
% \textsc{CTEX.ORG}.
% \newblock \textit{\CTeX{} 宏集手册} [EB/OL].
% \newblock version 2.4.14,
% \newblock (2018-05-02) 
% \urlprefix\url{https://ctan.org/pkg/ctex}
% \urlprefix\CTANurl[文档及源代码:]{language/chinese/ctex/ctex.pdf}
%
% \bibitem{xeCJK}
% \textsc{CTEX.ORG}.
% \newblock \textit{\pkg{xeCJK} 宏包} [EB/OL].
% \newblock version 3.7.1,
% \newblock (2018-04-30) 
% \urlprefix\url{https://ctan.org/pkg/xecjk}
% \urlprefix\CTANurl[文档及源代码:]{macros/xetex/latex/xecjk/xeCJK.pdf}
%
% \bibitem{natbib}
% \textsc{Daly P W}.
% \newblock \textit{Natural Sciences Citations and References} [EB/OL].
% \newblock version 8.31b,
% \newblock (2010-09-13) 
% \urlprefix\url{https://ctan.org/pkg/natbib}
% \urlprefix\CTANurl[文档及源代码:]{macros/latex/contrib/natbib/natbib.pdf}
%
% \bibitem{source3}
% \textsc{The \LaTeX3 Project}.
% \newblock \textit{The \LaTeX3 Sources} [CP/OL].
% \newblock (2019-03-05) 
% \urlprefix\url{https://ctan.org/pkg/l3kernel}
% \urlprefix\CTANurl[源代码:]{macros/latex/contrib/l3kernel/source3.pdf}
%
% \bibitem{biblatex}
% \textsc{Lehman P}, \textsc{Kime P}, \textsc{Boruvka A}, et al.
% \newblock \textit{The \pkg{biblatex} Package} [EB/OL].
% \newblock version 3.12,
% \newblock (2018-10-18) 
% \urlprefix\url{https://ctan.org/pkg/biblatex}
% \urlprefix\CTANurl[文档:]{macros/latex/contrib/biblatex/doc/biblatex.pdf}
%
% \bibitem{lshort}
% \textsc{Oetiker T}, \textsc{Partl H}, \textsc{Hyna I}, et al.
% \newblock \textit{The Not So Short Introduction to \LaTeXe{}: Or \LaTeXe{} in 139 minutes} [EB/OL].
% \newblock version 6.2,
% \newblock (2018-02-28) 
% \urlprefix\url{https://ctan.org/pkg/lshort-english}
% \urlprefix\CTANurl[文档:]{info/lshort/english/lshort.pdf}
%
% \bibitem{lshort-zh-cn}
% \textsc{Oetiker T}, \textsc{Partl H}, \textsc{Hyna I}, et al.
% \newblock \textit{一份不太简短的 \LaTeXe{} 介绍: 或 106 分钟了解 \LaTeXe{}} [EB/OL].
% \newblock \CTeX{} 开发小组, 译.
% \newblock 原版版本 version 6.2, 中文版本 version 6.0,
% \newblock (2018-09-01) 
% \urlprefix\url{https://ctan.org/pkg/lshort-zh-cn}
% \urlprefix\CTANurl[文档:]{info/lshort/chinese/lshort-zh-cn.pdf}
%
% \bibitem{biblatex-gb7714-2015}
% 胡振震.
% \newblock \textit{符合 GB/T 7714-2015 标准的 biblatex 参考文献样式} [EB/OL].
% \newblock version 1.0q,
% \newblock (2019-02-11) 
% \urlprefix\url{https://ctan.org/pkg/biblatex-gb7714-2015}
% \urlprefix\CTANurl[文档:]{biblatex-contrib/biblatex-gb7714-2015/biblatex-gb7714-2015.pdf}
%
% \bibitem{gbt7714}
% 李泽平(\textsc{Zeping L}).
% \newblock \textit{GB/T 7714-2015 \BibTeX{} Style} [EB/OL].
% \newblock version 1.0.9,
% \newblock (2018-08-05) 
% \urlprefix\url{https://ctan.org/pkg/gbt7714}
% \urlprefix\CTANurl[文档:]{biblio/bibtex/contrib/gbt7714/gbt7714.pdf}
%
% \bibitem{cquthesis}
% 李振楠.
% \newblock \textit{\textsc{CquThesis}:重庆大学毕业论文 \LaTeX{} 模板} [EB/OL].
% \newblock version 1.30,
% \newblock (2018-02-23) 
% \urlprefix\url{https://ctan.org/pkg/cquthesis}
% \urlprefix\CTANurl[文档及源代码:]{macros/latex/contrib/cquthesis/cquthesis.pdf}
%
% \bibitem{thuthesis}
% 薛瑞尼.
% \newblock \textit{\textsc{ThuThesis}:清华大学学位论文模板} [EB/OL].
% \newblock version 5.4.5,
% \newblock (2018-05-17) 
% \urlprefix\url{https://ctan.org/pkg/thuthesis}
% \urlprefix\CTANurl[文档及源代码:]{macros/latex/contrib/thuthesis/thuthesis.pdf}
%
% \emph{以下模版未收录至 CTAN,但仍然保持活跃更新。}
%
% \bibitem{sjtuthesis}
% \textsc{SJTUG}.
% \newblock \textit{上海交通大学 \XeLaTeX{} 学位论文及课程论文模板} [EB/OL].
% \newblock version 0.10.2,
% \newblock (2018-11-05)
% \urlprefix\url{https://github.com/sjtug/SJTUThesis}
%
% \bibitem{ustcthesis}
% \textsc{USTC \TeX{} User Group}.
% \newblock \textit{中国科学技术大学学位论文 \LaTeX{} 模板} [EB/OL].
% \newblock version 3.1.03,
% \newblock (2019-01-01)
% \urlprefix\url{https://github.com/ustctug/ustcthesis}
%
% \bibitem{ucasthesis}
% \textsc{mohuangrui}.
% \newblock \textit{\pkg{ucasthesis} 国科大学位论文 \LaTeX{} 模板} [EB/OL].
% \newblock (2019-03-14)
% \urlprefix\url{https://github.com/mohuangrui/ucasthesis}
%
% \emph{以下模版现已停止更新。}
%
% \bibitem{pandoxie2014fduthesislatex}
% \textsc{Pandoxie}.
% \newblock \textit{Fudan University-Latex Template} [EB/OL].
% \newblock (2014-06-07) 
% \urlprefix\url{https://github.com/Pandoxie/FDU-Thesis-Latex}
%
% \bibitem{richard2016fudanthesis}
% \textsc{richard}.
% \newblock \textit{复旦大学硕士学位论文模板} [EB/OL].
% \newblock (2016-01-31) 
% \urlprefix\url{https://github.com/richarddzh/fudan-thesis}
%
% \bibitem{数院毕业论文格式}
% 复旦大学数学科学学院.
% \newblock \textit{毕业论文格式 tex 版和 word 版} [EB/OL].
% \urlprefix\url{http://math.fudan.edu.cn/show.aspx?info_lb=664&flag=101&info_id=1816}
%
% \bibitem{数院毕业论文格式更新}
% 复旦大学数学科学学院.
% \newblock \textit{毕业论文格式: Word、\TeX{} 模板更新} [EB/OL].
% \urlprefix\url{http://math.fudan.edu.cn/Show.aspx?info_lb=664&info_id=1855&flag=101}
%
% \subsection{其他}
%
% \bibitem{wright2009dtxfile}
% \textsc{Wright J}.
% \newblock \textit{A model dtx file} [EB/OL].
% \newblock (2009-10-06) 
% \urlprefix\url{https://www.texdev.net/2009/10/06/a-model-dtx-file/}
%
% \bibitem{孔雀计划}
% 刘庆(\textsc{Eric Q L}).
% \newblock \textit{孔雀计划:中文字体排印的思路} [EB/OL].
% \urlprefix\url{https://thetype.com/kongque/}
%
% \end{thebibliography}
%
% \clearpage
%
%^^A! \end{document}
%
% \end{documentation}
