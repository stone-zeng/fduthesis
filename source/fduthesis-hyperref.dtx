% \subsection{\pkg{hyperref} 相关配置}
%
% \changes{v0.4}{2017/08/13}{新增 \pkg{hyperref} 相关配置,包括超链接
%   样式(\opt{style/hyperlink} 与 \opt{style/hyperlink-color} 选项)
%   及 PDF 元信息等。}
% \changes{v0.7}{2018/01/23}{将 \pkg{hyperref} 相关配置移至模板末尾,
%   以减少冲突。}
%
% \begin{macro}{\hypersetup,\fdu_hyperref_setup:n}
% \pkg{hyperref} 宏包是在导言区之后才引入的。若要在导言区中使用
% \tn{hypersetup} 命令,必须另行定义。
%    \begin{macrocode}
\NewDocumentCommand \hypersetup { m }
  { \fdu_hyperref_setup:n {#1} }
\cs_new_protected:Npn \fdu_hyperref_setup:n #1
  { \clist_gput_right:Nn \g_@@_to_hyperref_clist {#1} }
%    \end{macrocode}
% \end{macro}
%
% \begin{macro}{\@@_set_hyperlink_color_key:n}
% 设置超链接颜色选项。最后的逗号用于确保 \pkg{l3keys} 可以正确解析,不能省去。
%    \begin{macrocode}
\cs_new:Npn \@@_set_hyperlink_color_key:n #1
  {
    hyperlink-color / \clist_item:nn {#1} {1} .code:n =
      {
        \@@_define_hyperlink_color:nnn
          { \clist_item:nn {#1} {2} }
          { \clist_item:nn {#1} {3} }
          { \clist_item:nn {#1} {4} }
        \fdu_hyperref_setup:n
          {
            linkcolor = fdu@link, linkbordercolor = fdu@link,
            urlcolor  = fdu@url,  urlbordercolor  = fdu@url,
            citecolor = fdu@cite, citebordercolor = fdu@cite
          }
      },
  }
%    \end{macrocode}
% \end{macro}
%
% \begin{macro}{\@@_define_hyperlink_color:nnn}
% 定义超链接颜色。
%    \begin{macrocode}
\cs_new_protected:Npn \@@_define_hyperlink_color:nnn #1#2#3
  {
    \definecolorset { HTML } { fdu@ } { }
      { link, #1; url, #2; cite, #3 }
  }
%    \end{macrocode}
% \end{macro}
%
%    \begin{macrocode}
\keys_define:nx { fdu / style }
  {
%    \end{macrocode}
%
% \begin{macro}{style/hyperlink}
% 超链接样式。
%    \begin{macrocode}
    hyperlink .choice:,
    hyperlink .value_required:n = true,
    hyperlink / border .code:n = { },
    hyperlink / color  .code:n =
      { \fdu_hyperref_setup:n { colorlinks = true } },
    hyperlink / none   .code:n =
      { \fdu_hyperref_setup:n { hidelinks  = true } },
%    \end{macrocode}
% \end{macro}
%
% \begin{macro}{style/hyperlink-color}
% 超链接颜色。
%    \begin{macrocode}
    hyperlink-color .choice:,
    hyperlink-color .value_required:n = true,
    \clist_map_function:nN
      {
        { autumn,    D70000, D75F00, AF8700 },
        { business,  D14542, 295497, 1F6E43 },
        { classic,   FF0000, 0000FF, 00FF00 },
        { default,   990000, 0000B2, 007F00 },
        { elegant,   961212, C31818, 9B764F },
        { fantasy,   FF4A19, FF3F94, 934BA1 },
        { material,  E91E63, 009688, 4CAF50 },
        { science,   CA0619, 389F9D, FF8920 },
        { summer,    00AFAF, 5F5FAF, 5F8700 },
        { graylevel, 616161, 616161, 616161 },
        { prl,       2D3092, 2D3092, 2D3092 }
      }
      \@@_set_hyperlink_color_key:n
  }
%    \end{macrocode}
% \end{macro}
%
% \changes{v0.6}{2017/10/29}{优化 URL 断行设置。}
%
% \begin{macro}{\fdu_allow_url_break:,\@@_add_url_break_points:}
% 允许 URL 在字母、数字和一些特殊符号处断行。见
% \url{https://bit.ly/2hhIjLW}。
%    \begin{macrocode}
\cs_new:Npn \fdu_allow_url_break:
  {
    \cs_new:Npn \@@_add_url_break_points:
      { \tl_map_function:NN \c_@@_url_break_points_tl \do }
    \@@_appto_cmd:Nn \UrlBreaks
      { \UrlOrds \@@_add_url_break_points: }
  }
%    \end{macrocode}
% \end{macro}
%
% \begin{variable}{\c_@@_url_break_points_tl}
% 额外的断行位置是 26 个英文字母(大小写)以及 10 个阿拉伯数字。
% \pkg{url} 提供的宏 \tn{UrlBreaks} 还包含了特殊符号 |*|、|-|、
% |~|、|'|、|"|、|-|,也被设置为允许断行。
%    \begin{macrocode}
\tl_const:Nn \c_@@_url_break_points_tl
  {
    abcdefghijklmnopqrstuvwxyz
    ABCDEFGHIJKLMNOPQRSTUVWXYZ
    0123456789
  }
%    \end{macrocode}
% \end{variable}
%
% \changes{v0.7e}{2019/04/23}{处理 \pkg{hyperref} 与 \pkg{unicode-math} 的兼容性问题。} 
%
% 在导言区末尾引入 \pkg{hyperref} 宏包。
%    \begin{macrocode}
\ctex_at_end_preamble:n
  {
    \RequirePackage { hyperref }
%    \end{macrocode}
% 此后 \tn{hypersetup} 命令由 \pkg{hyperref} 宏包接管。
%    \begin{macrocode}
    \hypersetup
      {
        bookmarksnumbered = true,
        psdextra          = true,
        unicode           = true,
%    \end{macrocode}
% 填写 PDF 元信息。
%    \begin{macrocode}
%<*class>
        pdftitle    = \l_@@_info_title_tl,
        pdfauthor   = \l_@@_info_author_tl,
        pdfkeywords = \l_@@_info_keywords_clist,
%</class>
%<*class-en>
        pdftitle    = \l_@@_info_title_en_tl,
        pdfauthor   = \l_@@_info_author_en_tl,
        pdfkeywords = \l_@@_info_keywords_en_clist,
%</class-en>
%       pdfsubject  = ,
        pdfcreator  = \c_@@_name_pdf_creator_tl
      }
%    \end{macrocode}
% 将导言区中通过 \cs{fdu_hyperref_setup:n} 进行的设置传入
% \tn{hypersetup}。
%    \begin{macrocode}
    \exp_args:NV \hypersetup \g_@@_to_hyperref_clist
%    \end{macrocode}
% URL 断行处理。
%    \begin{macrocode}
    \fdu_allow_url_break:
%    \end{macrocode}
% 手动开启 \pkg{biblatex} 的 \pkg{hyperref} 支持。
%    \begin{macrocode}
    \bool_if:NF \l_@@_bibtex_bool { \BiblatexManualHyperrefOn }
  }
%    \end{macrocode}
%
% 在 PDF 字符串中设置 \tn{fdu@kai} 命令为空,以抑制 \pkg{hyperref}
% 的警告信息。
%    \begin{macrocode}
\ctex_at_end_package:nn { hyperref }
  {
    \pdfstringdefDisableCommands
      {
        \cs_set_eq:NN \fdu@kai \prg_do_nothing:
        \cs_set_eq:NN \quad    \c_space_tl
        \cs_set_eq:NN \qquad   \c_space_tl
      }
  }
%    \end{macrocode}
%
