% \iffalse meta-comment
%
% Copyright (C) 2017--2019 by Xiangdong Zeng <xdzeng96@gmail.com>
%
% This work may be distributed and/or modified under the conditions of the
% LaTeX Project Public License, either version 1.3c of this license or (at
% your option) any later version. The latest version of this license is in:
%
%   http://www.latex-project.org/lppl.txt
%
% and version 1.3 or later is part of all distributions of LaTeX version
% 2005/12/01 or later.
%
% This work has the LPPL maintenance status `maintained'.
%
% The Current Maintainer of this work is Xiangdong Zeng.
%
% \fi
%
% \begin{implementation}
%
%    \begin{macrocode}
% Template
\DeclareObjectType { fdu / layout } { \c_zero_int }

\DeclareTemplateInterface { fdu / layout } { main } { \c_zero_int }
  {
    geometry : code =
      {
        paper      = a4paper,
        vmargin    = 2.54 cm,
        hmargin    = 3.18 cm,
        headheight = 15 pt,
        % showframe  = true,
      },

    page-style-twoside : code =
      {
        \fancyhf { }
        \fancyhead [ EL ] { \small \nouppercase { \fdu@kai \leftmark  } }
        \fancyhead [ OR ] { \small \nouppercase { \fdu@kai \rightmark } }
        \fancyfoot [ C  ] { \small \thepage }
      },
    page-style-oneside : code =
      {
        \fancyhf { }
        \fancyhead [ C ] { \small \nouppercase { \fdu@kai \leftmark } }
        \fancyfoot [ C ] { \small \thepage }
      },
    page-style-chapter-twoside : code =
      {
        \fancyhf { }
        \fancyfoot [ C ] { \small \thepage }
        \cs_set:Npn \headrulewidth { 0 pt }
      },
    page-style-chapter-oneside : code =
      \KeyValue { page-style-chapter-twoside },

    chapter : code =
      {
        format      = \huge \normalfont \sffamily \centering,
        beforeskip  = 50 pt,
        afterskip   = 40 pt,
        number      = \@@_arabic:n { chapter },
        fixskip     = true
      },
    section : code =
      {
        format      = \Large \normalfont \sffamily \raggedright,
        beforeskip  = 3.5 ex plus 1.0 ex minus 0.2 ex,
        afterskip   = 2.7 ex plus 0.5 ex,
        fixskip     = true
      },
    subsection : code =
      {
        format      = \large \normalfont \sffamily \raggedright,
        beforeskip  = 3.25 ex plus 1.0 ex minus 0.2 ex,
        afterskip   = 2.5  ex plus 0.3 ex,
        fixskip     = true
      },
    subsubsection : code,
    paragraph : code,
    subparagraph : code,

    figure-caption : code =
      {
        font     = small,
        labelsep = quad
      },
    table-caption : code =
      {
        font     = { small, sf },
        labelsep = quad
      },
    algorithm-caption : code,
    figure-number : code =
      { \@@_arabic:n { chapter } - \@@_arabic:n { figure } },
    table-number  : code =
      { \@@_arabic:n { chapter } - \@@_arabic:n { table  } },
    algorithm-number  : code,
  }

\cs_new_protected:Npn \@@_def_page_style:nn #1#2
  { \fancypagestyle { fdu-page-style-#1 } {#2} }

\DeclareTemplateCode { fdu / layout } { main } { \c_zero_int }
  {
    geometry                   = \geometry {#1},
    page-style-twoside         = \@@_def_page_style:nn { twoside         } {#1},
    page-style-oneside         = \@@_def_page_style:nn { oneside         } {#1},
    page-style-chapter-twoside = \@@_def_page_style:nn { chapter-twoside } {#1},
    page-style-chapter-oneside = \@@_def_page_style:nn { chapter-oneside } {#1},
    chapter                    = \ctexset { chapter       = {#1} },
    section                    = \ctexset { section       = {#1} },
    subsection                 = \ctexset { subsection    = {#1} },
    subsubsection              = \ctexset { subsubsection = {#1} },
    paragraph                  = \ctexset { paragraph     = {#1} },
    subparagraph               = \ctexset { subparagraph  = {#1} },
    figure-caption             = \captionsetup {#1},
    table-caption              = \captionsetup {#1},
    algorithm-caption          = \captionsetup {#1},
    figure-number              = \cs_set:Npn \thefigure    {#1},
    table-number               = \cs_set:Npn \thetable     {#1},
    algorithm-number           = \cs_set:Npn \thealgorithm {#1},
  }
  {
    \AssignTemplateKeys
    \cs_show:N \fdu_if_draft:TF
    \pagestyle { fdu-page-style-twoside }
    \keys_set:nn { ctex } { chapter / pagestyle = fdu-page-style-chapter-twoside }
  }

\DeclareInstance { fdu / layout } { default } { main } { }
\UseInstance { fdu / layout } { default }
\debug_off:n { all }

%    \end{macrocode}
%
%^^A % \section{页面布局}
%^^A %
%^^A % 利用 \pkg{geometry} 宏包设置纸张大小、页面边距以及页眉高度。这里,
%^^A % $\SI{2.54}{\centi\meter}=\SI{1}{in}$,
%^^A % $\SI{3.18}{\centi\meter}=\SI{1.25}{in}$。
%^^A %    \begin{macrocode}
%^^A \geometry
%^^A   {
%^^A     paper      = a4paper,
%^^A     vmargin    = 2.54 cm,
%^^A     hmargin    = 3.18 cm,
%^^A     headheight = 15 pt
%^^A   }
%^^A %    \end{macrocode}
%^^A %
%^^A % 草稿模式下显示页面边框及页眉、页脚线 。
%^^A %    \begin{macrocode}
%^^A \bool_if:NT \g_@@_draft_bool { \geometry { showframe } }
%^^A %    \end{macrocode}
%^^A %
%^^A % \section{页眉页脚}
%^^A %
%^^A % 清除默认页眉页脚格式。
%^^A %    \begin{macrocode}
%^^A \fancyhf { }
%^^A %    \end{macrocode}
%^^A %
%^^A % \begin{variable}{\l_@@_header_center_mark_tl}
%^^A % 保存中间页眉的文字。正文中设置为空,目录、摘要、符号表等设置为相应标题。
%^^A %    \begin{macrocode}
%^^A \tl_new:N \l_@@_header_center_mark_tl
%^^A %    \end{macrocode}
%^^A % \end{variable}
%^^A %
%^^A % 构建页眉,要在单面或双面下分别设置。
%^^A %
%^^A % \cs{fancyhead} 的选项中,\opt{E} 和 \opt{O} 分别表示偶数(even)
%^^A % 和奇数(odd), 而 \opt{L}、\opt{R} 和 \opt{C} 则分别表示左
%^^A % (left)、右(right)和中间(center)。按照通常的排版规则,
%^^A % 在双面模式下,偶数页的中间页眉文字在左,奇数页则在右。单面模式下,
%^^A % 左右页眉都要显示。
%^^A %    \begin{macrocode}
%^^A \bool_if:NTF \g_@@_twoside_bool
%^^A %<*class>
%^^A   {
%^^A     \fancyhead [ EL ] { \small \nouppercase { \fdu@kai \leftmark  } }
%^^A     \fancyhead [ OR ] { \small \nouppercase { \fdu@kai \rightmark } }
%^^A   }
%^^A   {
%^^A     \fancyhead [ L ] { \small \nouppercase { \fdu@kai \leftmark  } }
%^^A     \fancyhead [ R ] { \small \nouppercase { \fdu@kai \rightmark } }
%^^A     \fancyhead [ C ]
%^^A       {
%^^A         \small \nouppercase
%^^A           { \fdu@kai \l_@@_header_center_mark_tl }
%^^A       }
%^^A   }
%^^A %</class>
%^^A %<*class-en>
%^^A   {
%^^A     \fancyhead [ EL ] { \small \nouppercase { \itshape \leftmark  } }
%^^A     \fancyhead [ OR ] { \small \nouppercase { \itshape \rightmark } }
%^^A   }
%^^A   {
%^^A     \fancyhead [ L ] { \small \nouppercase { \itshape \leftmark  } }
%^^A     \fancyhead [ R ] { \small \nouppercase { \itshape \rightmark } }
%^^A     \fancyhead [ C ]
%^^A       {
%^^A         \small \nouppercase
%^^A           { \itshape \l_@@_header_center_mark_tl }
%^^A       }
%^^A   }
%^^A %</class-en>
%^^A %    \end{macrocode}
%^^A %
%^^A % 构建页脚,用来显示页码。选项 \opt{C} 表示居中(center)。
%^^A %    \begin{macrocode}
%^^A \fancyfoot [ C ] { \small \thepage }
%^^A %    \end{macrocode}
%^^A %
%^^A % 关闭横线显示(未启用)。
%^^A %    \begin{macrocode}
%^^A % \RenewDocumentCommand \headrulewidth { } { 0 pt }
%^^A %    \end{macrocode}
%^^A %
%^^A % \begin{macro}{\cleardoublepage}
%^^A % 重定义 \tn{cleardoublepage},使得偶数页面在没有内容时也不显示页眉页脚,见
%^^A % \url{https://tex.stackexchange.com/a/1683}。最后清空中间页眉,确保正文部分
%^^A % 页眉显示正确。
%^^A %    \begin{macrocode}
%^^A \RenewDocumentCommand \cleardoublepage { }
%^^A   {
%^^A     \clearpage
%^^A     \bool_if:NT \g_@@_twoside_bool
%^^A       {
%^^A         \int_if_odd:nF \c@page
%^^A           { \hbox:n { } \thispagestyle { empty } \newpage }
%^^A       }
%^^A     \tl_gset:Nn \l_@@_header_center_mark_tl { }
%^^A   }
%^^A %    \end{macrocode}
%^^A % \end{macro}
%^^A %
%^^A % \pkg{ctex} 宏包使用 \opt{heading} 选项后,会把页面格式设置为 |headings|。
%^^A % 因此必须在 \pkg{ctex} 调用之后重新设置 \cs{pagestyle} 为 |fancy|。
%^^A %    \begin{macrocode}
%^^A \pagestyle { fancy }
%^^A %    \end{macrocode}
%^^A %
%^^A % \section{章节标题结构}
%^^A %
%^^A % |\keys_set:nn{ctex}| 实际相当于 \cs{ctexset}。
%^^A %    \begin{macrocode}
%^^A \keys_set:nn { ctex }
%^^A   {
%^^A %    \end{macrocode}
%^^A % 设置章(chapter)、节(section)与小节(sub-section)标题样式。
%^^A % 此处使用 \kvopt{fixskip}{true} 选项来抑制前后的多余间距。
%^^A %    \begin{macrocode}
%^^A     chapter =
%^^A       {
%^^A %<class>        format      = \huge \normalfont \sffamily \centering,
%^^A %<*class-en>
%^^A         format      = \centering,
%^^A         nameformat  = \LARGE \bfseries,
%^^A         titleformat = \huge \bfseries,
%^^A         aftername   = \par \nobreak \vskip 10 pt,
%^^A %</class-en>
%^^A         beforeskip  = 50 pt,
%^^A         afterskip   = 40 pt,
%^^A         number      = \@@_arabic:n { chapter },
%^^A         fixskip     = true
%^^A       },
%^^A     section =
%^^A       {
%^^A %<class>        format      = \Large \normalfont \sffamily \raggedright,
%^^A %<class-en>        format      = \Large \bfseries \raggedright,
%^^A         beforeskip  = 3.5 ex plus 1.0 ex minus 0.2 ex,
%^^A         afterskip   = 2.7 ex plus 0.5 ex,
%^^A         fixskip     = true
%^^A       },
%^^A     subsection =
%^^A       {
%^^A %<class>        format      = \large \normalfont \sffamily \raggedright,
%^^A %<class-en>        format      = \large \bfseries \raggedright,
%^^A         beforeskip  = 3.25 ex plus 1.0 ex minus 0.2 ex,
%^^A         afterskip   = 2.5  ex plus 0.3 ex,
%^^A         fixskip     = true
%^^A       }
%^^A   }
%^^A %    \end{macrocode}
%^^A %
% \changes{v0.7d}{2019/03/24}{优化目录、摘要、参考文献等的标题实现。}
%
% \begin{macro}{\@@_chapter:n,\@@_chapter:V}
% 手动生成章的标题,用于摘要、参考文献等。
%    \begin{macrocode}
\cs_new_protected:Npn \@@_chapter:n #1
  {
    \group_begin:
      \ctexset { chapter / numbering = false }
      \chapter {#1}
      \@@_chapter_header:n {#1}
    \group_end:
  }
\cs_generate_variant:Nn \@@_chapter:n { V }
%    \end{macrocode}
% \end{macro}
%
% \begin{macro}{\@@_chapter_no_toc:n,\@@_chapter_no_toc:V}
% 目录自身不出现在目录中,需特别处理。参考
% \url{https://tex.stackexchange.com/a/1821}。
%    \begin{macrocode}
\cs_new_protected:Npn \@@_chapter_no_toc:n #1
  {
    \chapter *           {#1}
    \@@_chapter_header:n {#1}
    \pdfbookmark [0] {#1} { toc }
  }
\cs_generate_variant:Nn \@@_chapter_no_toc:n { V }
%    \end{macrocode}
% \end{macro}
%
% \begin{macro}{\@@_chapter_header:n}
% 单页模式下,目录、摘要、符号表等的页眉中间为相应标题,左右为空。
%    \begin{macrocode}
\cs_new_protected:Npn \@@_chapter_header:n #1
  {
    \bool_if:NTF \g_@@_twoside_bool
      { \markboth {#1} {#1} }
      {
        \markboth { } { }
        \tl_gset:Nn \l_@@_header_center_mark_tl {#1}
      }
  }
%    \end{macrocode}
% \end{macro}
%^^A %
%^^A % \section{图表绘制;浮动体}
%^^A %
%^^A % \changes{v0.3}{2017/07/09}{支持浮动体。}
%^^A %
%^^A % 分别设置浮动体 \env{figure} 和 \env{table} 的标题样式。
%^^A %    \begin{macrocode}
%^^A \captionsetup [ figure ]
%^^A   {
%^^A     font     = small,
%^^A     labelsep = quad
%^^A   }
%^^A \captionsetup [ table  ]
%^^A   {
%^^A     font     = { small, sf },
%^^A     labelsep = quad
%^^A   }
%^^A %    \end{macrocode}
%^^A %
%^^A % \begin{macro}{\thefigure,\thetable}
%^^A % \changes{v0.7}{2018/01/17}{改为可完全展开的命令。}
%^^A % 重定义图表编号。
%^^A %    \begin{macrocode}
%^^A \cs_set:Npn \thefigure
%^^A   { \@@_arabic:n { chapter } - \@@_arabic:n { figure } }
%^^A \cs_set:Npn \thetable
%^^A   { \@@_arabic:n { chapter } - \@@_arabic:n { table  } }
%^^A %    \end{macrocode}
%^^A % \end{macro}
%
% \end{implementation}
%
