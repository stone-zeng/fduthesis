\PassOptionsToPackage{log-declarations=false}{xparse}
%\documentclass{article}
%\usepackage{fontspec}
%\setmainfont{TeX Gyre Cursor}[Ligatures=NoCommon]
%
%\begin{document}
%Hello, world!
%\textbf{Hello, world!}
%\textit{Hello, world!}
%\textsf{Hello, world!}
%\end{document}

% \documentclass[twoside,nofonts]{fduthesis}
\documentclass{ctexbook}
\usepackage{kantlipsum}
\usepackage{zhlipsum}
\usepackage{unicode-math}

% \fdusetup{
%   style = {
%     % font = libertinus,
%     % cjkfont = founder,
%     fullwidthstop,
%     footnotestyle = plain,
%     % automakecover = false
%   },
%   info = {
%     title = {自河南经乱关内阻饥兄弟离散各在一处},
%     title* = {Measurements of the interaction between energetic
%       photons and hadrons show that the interaction},
% %    date = {2017年2月10日},
%     author = {某某某},
%     author* = {Xiangdong Zeng},
%     supervisor = {陈丙丁 \quad 教授},
%     instructors = {
%       张五六 \quad 工程师,
%       赵\quad 甲 \quad 工程师,
%       王三四 \quad 讲\quad 师
%     },
%     major = {物理学},
%     department = {凝聚态物理系},
%     secretlevel = ii,
%     secretyear = {三年},
%     studentid = {14307110000},
%     keywords = {\LaTeX, hih, vwvs, 物理, 中心法则, vsbd, 为侨服务},
%     keywords* = {\LaTeX, hih, vwvs, Physics, Central Rule, vsbd},
%     clc = {O414.1/65}
%   }
% }

\setmainfont{XITS}
\setsansfont{TeX Gyre Heros}
\setmonofont{texgyrecursor-regular.otf}%[Ligatures=NoCommon]
\setmathfont{xits-math.otf}

\setCJKmainfont[ItalicFont=FandolFang]{FandolSong}
\setCJKsansfont{FandolHei}

\newcommand{\fonttesttext}{你好,世界\symbol{12290}Hello, world!}
\newcommand\fonttest{%
  正常:\qquad\fonttesttext \par
  粗体:\qquad\textbf{\fonttesttext} \par
  倾斜:\qquad\textit{\fonttesttext} \par
  小型大写:\qquad\textsc{\fonttesttext}
}

\begin{document}
\frontmatter
%\tableofcontents
%
%\begin{abstract}
%\emph{中心法则}形成了分子生物学的生命观:生命世界在信息和规律上是统一的;
%生命的物质基础和主宰物质是不同的,生命的物质基础是以核酸蛋白质整合
%体系为主宰的原生质各种必要的物质组分及其实在的相互作用;生命运动
%的本质在于组成生命的各物质之间以及生命的物质、能量与信息之间和生命
%自身与环境之间的不断的相互作用。不少人不少人,如果是到无穷大。
%\end{abstract}
%
%\begin{abstract*}
%\LaTeX3 does not use @ as a ``letter'' for defining internal
%macros. Instead, the symbols are used in internal macro names to
%provide structure. The name of each function is divided into
%logical units using separates the name of the function from the
%argument specifier (``arg-spec''). This describes the arguments
%expected by the function.
%\end{abstract*}

\mainmatter
\chapter{文本,字体,脚注 \quad Text, font and footnote}
\section{文字与段落 Text and paragraph}
\subsection{中文文本 Chinese}
\zhlipsum

\subsection{英文文本 English}
\kant

\clearpage

\section{字体 Font}
\subsection{普通字体 Roman}
\fonttest

\subsection{无衬线字体 Sans-serif}
\textsf{\fonttest}

\subsection{打字机字体 Typewriter}
\texttt{\fonttest}

\subsection{句号}
如果\symbol{"002E}会突然:\textsc{Full Stop} \par
如果\symbol{"3002}会突然:\textsc{Ideographic Full SStop} \par
如果\symbol{"FF0E}会突然:\textsc{Fullwidth Full Stop} \par
如果\symbol{"FF61}会突然:\textsc{Halfwidth Ideographic Full Stop} \par

\clearpage

\section{脚注 Footnote}
脚注\footnote{脚注1如果会突然}。

脚注 \footnote{脚注2。千千万的。}。未取得的

脚注。\footnote{脚注3是一个长脚注。\zhlipsum*[2]}未取得的

\textit{脚注倾斜。 \footnote{脚注4}未取得的}

\textbf{脚注加粗。\footnote{脚注5} 未取得的}

vs\footnote{脚注6}未取得的

vs \footnote{脚注7要分段。\par 不舒服不得不运河滩上野跑,头顶着毒热的阳光,身上再裹起兜肚,一不风凉,
二又窝汗,穿不了一天,就得起大半身痱子。再有,全村跟他一般大的小姑娘,
谁的兜肚也没有这么花儿草儿的鲜艳,他穿在身上,男不男,女不女,
小姑娘们要用手指刮破脸蛋儿。}未取得的

vs\footnote{脚注8}ge

vs\footnote{脚注9} jsty

vs\footnote{脚注10分三段。青大娘大高个儿,一双大脚,青铜肤色,
嗓门也亮堂,骂起人来,方圆二三十里,敢说找不出能够招架几个回合的敌手。
一丈青大娘骂人,就像雨打芭蕉,长短句。\par
青大娘大高个儿,一双大脚,青铜肤色,
嗓门也亮堂,骂起人来,方圆二三十里,敢说找不出能够招架几个回合的敌手。
一丈青大娘骂人,就像雨打芭蕉,长短句。 \par
青大娘大高个儿,一双大脚,青铜肤色,
嗓门也亮堂,骂起人来,方圆二三十里,敢说找不出能够招架几个回合的敌手。
一丈青大娘骂人,就像雨打芭蕉,长短句。}

\chapter{数学与定理}
\section{数学 Math}
\[\pi=\sb{33}\]

\[
  \int\sin x\,\mathrm{d}x=\cos x + C
\]

% \begin{proof}
% 道千乘之国,敬事而信,节用而爱人,使民以时。
% \end{proof}

% \begin{definition}
% 证明完毕/证讫,又写作Q.E.D.。这是拉丁词组“quod erat demonstrandum”(这就是所要证明的)的缩写,译自希腊语“ὅπερ ἔδει δεῖξαι”(hoper edei deixai),很多早期数学家用过,包括欧几里得和阿基米德。“Q.E.D.”可以在证明的尾段写出,以显示证明所需的结论已经完整了。
% \end{definition}

% \begin{lemma}
%   这是一条华丽丽的引理。
% \end{lemma}

% \begin{proof}
% 先帝创业未半而中道崩殂,今天下三分,益州疲弊,此诚危急存亡之秋也。
% \begin{equation}
%   \sum_{k=0}^{\infty} \frac{1}{x^k} = \int \sin x dx
% \end{equation}
% \end{proof}

% \begin{proof}
% 先帝创业未半而中道崩殂,今天下三分,益州疲弊,此诚危急存亡之秋也。
% \begin{equation*}
%   \sum_{k=0}^{\infty} \frac{1}{x^k} = \int \sin x dx
% \end{equation*}
% \end{proof}

% \begin{lemma}
%   这又是一条华丽丽的引理。
% \end{lemma}

% \chapter{图表 vs 浮动体}
% \section{title}
% Myriad,英语单词,意为「无数的」。同时,「Myriad」也是一款字体的名字。
% 由罗伯特·斯林巴赫(Robert Slimbach,1956年-)和卡罗·图温布利
% (Carol Twombly,1959年-)在1990年到1992年期间以 Frutiger 字体为蓝本
% 为 Adobe 公司设计。 Myriad 是早期数码字体时代的先驱,伴随着技术的成长
% 一路走来。

% \begin{figure}[h]
%   \centering
%   \includegraphics[width=3cm]{../logo/fudan-emblem.pdf}
%   \includegraphics[width=4cm]{../logo/fudan-emblem.pdf}
%   \caption{Multiple Master 是 Type 1字体格式的扩展部分。Type 1 是利用
%     PostScript 语言描述字形信息的字体系统。Type 1字体是第一款矢量字体
%     (outline font),通过二维坐标系中的关键点和三次贝塞尔曲线描述字体
%     的边缘,在屏幕显示和输出时,在光栅图像处理器内,根据字号大小计算
%     出字体边缘(栅格化)。}
% \end{figure}

% 如今,它更多地和我们相见在显示屏幕上。当然,还有那著名的标榜设计的
% 电子品牌。1992 年,耗时两年开发的 Myriad 终于发布了历史上第一个版本:
% Myriad MM。

% \section{title}
% 这款温和且具有良好可读性的人文主义无衬线字体,集诸多当时最新的数字
% 字体技术于一身。 后缀 MM,意为 Multiple Master,没有找到对应的中文
% 译名,我们权且称之为「多母板技术」。Myriad 是最早采用 Multiple Master
% 技术的无衬线字体之一。这项技术的原理是在坐标轴(Axis)的区间两端设计
% 极限母板,中间的变量则采取线性或非线性变化,对于字体来说,字型的宽度、
% 粗细甚至有无衬线,都可以在坐标轴上设置。此外,MM 技术还提供了在小字号
% 下屏幕显示的视觉修正(Optical Adjustment),也就是说,同一款字体,在
% 小字号时,其字间距和笔画粗细,会被适当地放大。而衬线字体,随着字号的
% 变小,衬线会相对变粗。视觉修正可以提高小字号字体的识别性,对于远低于
% 印刷分辨率的电脑屏幕来说,也具有重要意义。

% \begin{table}[h]
%   \centering
%   \caption{一个 normal 表格}
%   \begin{tabular}{ccc}
%     \hline
%     \bfseries 功能 & \bfseries 环境 & \bfseries code \\
%     \hline
%     表格 & tabular & \ttfamily \backslash begin\{tabular\} ... \backslash end\{tabular\} \\
%     插图 & figure  & \ttfamily \backslash begin\{figure\}  ... \backslash end\{figure\}  \\
%     居中 & center  & \ttfamily \backslash begin\{center\}  ... \backslash end\{center\}  \\
%     \hline
%   \end{tabular}
% \end{table}

% 在 Multiple Master 的时代,字号是从6pt到72pt之间非线性设置的。这一传统
% 保留到了今天 Truetype 和 Opentype 的 Single Master 时代。Adobe 软件的
% 字体下拉菜单,仍然只显示6到72pt 的字号。

\end{document}
