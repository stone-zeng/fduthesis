\PassOptionsToPackage{log-declarations=false}{xparse}
\documentclass[twoside,nofonts]{fduthesis-en}
\usepackage{kantlipsum}

\fdusetup{
  style = {
    % font = libertinus,
    % cjkfont = founder,
    % fullwidthstop,
    % footnotestyle = pifont,
    % automakecover = false
  },
  info = {
    title = {自河南经乱关内阻饥兄弟离散各在一处},
    title* = {Measurements of the interaction between energetic
      photons and hadrons show that the interaction},
%    date = {2017年2月10日},
    author = {某某某},
    author* = {Xiangdong Zeng},
    supervisor = {陈丙丁 \quad 教授},
    instructors = {
      张五六 \quad 工程师,
      赵\quad 甲 \quad 工程师,
      王三四 \quad 讲\quad 师
    },
    major = {物理学},
    department = {凝聚态物理系},
    secretlevel = ii,
    secretyear = {三年},
    studentid = {14307110000},
    keywords = {\LaTeX, hih, vwvs, 物理, 中心法则, vsbd, 为侨服务},
    keywords* = {\LaTeX, hih, vwvs, Physics, Central Rule, vsbd},
    clc = {O414.1/65}
  }
}

\newcommand{\fonttesttext}{Hello, world! Hello, \TeX{}!}
\newcommand\fonttest{
  \begin{tabular}{lc}
    Normal:        & \fonttesttext \\
    Bold:          & \textbf{\fonttesttext} \\
    Italic:        & \textit{\fonttesttext} \\
    Small Capital: & \textsc{\fonttesttext}
  \end{tabular}
}

\begin{document}
\frontmatter
\tableofcontents

\begin{abstract}
\LaTeX3 does not use @ as a ``letter'' for defining internal
macros. Instead, the symbols are used in internal macro names to
provide structure. The name of each function is divided into
logical units using separates the name of the function from the
argument specifier (``arg-spec''). This describes the arguments
expected by the function.

\kant[1-5]
\end{abstract}

\begin{notation}
$\sin$      &  Sine \\
HPC         &  High Performance Computing \\
SMP         &  Symmetrical Multi-Processing \\
API         &  Application Programming Interface \\
PI          &  Polyimide \\
PBI         &  Polybenzimidazole \\
PY          &  Polypyrron \\
$\Delta G$  &  Activation Free Energy \\
$\chi$      &  Transmission Coefficient \\
$E$         &  Energy \\
$m$         &  Mass \\
$c$         &  Speed of Light \\
$P$         &  Possibility \\
$T$         &  Time \\
$v$         &  Velocity \\
Wikipedia   &  Wikipedia (/ˌwɪkᵻˈpiːdiə/ (About this sound listen) or /ˌwɪkiˈpiːdiə/ (About this sound listen) \emph{WIK-i-PEE-dee-ə}) is a free online encyclopedia with the aim to allow anyone to edit articles. Wikipedia is the largest and most popular general reference work on the Internet and is ranked among the ten most popular websites. Wikipedia is owned by the nonprofit Wikimedia Foundation. \\
$\sin$      &  Sine \\
HPC         &  High Performance Computing \\
SMP         &  Symmetrical Multi-Processing \\
API         &  Application Programming Interface \\
PI          &  Polyimide \\
PBI         &  Polybenzimidazole \\
PY          &  Polypyrron \\
$\Delta G$  &  Activation Free Energy \\
$\chi$      &  Transmission Coefficient \\
$E$         &  Energy \\
$m$         &  Mass \\
$c$         &  Speed of Light \\
$P$         &  Possibility \\
$T$         &  Time \\
$v$         &  Velocity \\
Wikipedia   &  Wikipedia (/ˌwɪkᵻˈpiːdiə/ (About this sound listen) or /ˌwɪkiˈpiːdiə/ (About this sound listen) \emph{WIK-i-PEE-dee-ə}) is a free online encyclopedia with the aim to allow anyone to edit articles. Wikipedia is the largest and most popular general reference work on the Internet and is ranked among the ten most popular websites. Wikipedia is owned by the nonprofit Wikimedia Foundation. \\
$\sin$      &  Sine \\
HPC         &  High Performance Computing \\
SMP         &  Symmetrical Multi-Processing \\
API         &  Application Programming Interface \\
PI          &  Polyimide \\
PBI         &  Polybenzimidazole \\
PY          &  Polypyrron \\
$\Delta G$  &  Activation Free Energy \\
$\chi$      &  Transmission Coefficient \\
$E$         &  Energy \\
$m$         &  Mass \\
$c$         &  Speed of Light \\
$P$         &  Possibility \\
$T$         &  Time \\
$v$         &  Velocity
\end{notation}

\mainmatter
\chapter{Text, font and footnote}
\section{Text and paragraph}
\subsection{English}
\kant

\clearpage

\section{Font}
\subsection{Roman}
\fonttest

\subsection{Sans-serif}
\textsf{\fonttest}

\subsection{Typewriter}
\texttt{\fonttest}

\clearpage

\section{Footnote}
Footnote\footnote{Religious influence had been strong in the Russian Empire.} 

Footnote \footnote{Religious influence had been strong in the Russian Empire.}. Hello

Footnote. \footnote{Footnote 3 is a long footnote. \kant[2]} Nothing

\textit{Italic footnote \footnote{Note 4}have nothing}

\textbf{Bold fotnote\footnote{Note 5} have nothing}

vs\footnote{Note 6}have nothing

vs \footnote{Note 7 has a par. \par The Russian Orthodox Church enjoyed
a privileged status as the church of the monarchy and took part in carrying
out official state functions. The immediate period following the establishment
of the Soviet state included a struggle against the Orthodox Church, which
the revolutionaries considered an ally of the former ruling classes.}have nothing

vs\footnote{Note 8}ge

vs\footnote{Note 9} jsty

vs\footnote{Note 10 has 3 par's.
The government encouraged a variety of trends. In art and literature, numerous
schools, some traditional and others radically experimental, proliferated. \par
The government encouraged a variety of trends. In art and literature, numerous
schools, some traditional and others radically experimental, proliferated. \par
The government encouraged a variety of trends. In art and literature, numerous
schools, some traditional and others radically experimental, proliferated.}

Text%
\footnote{This is a footnote.}%
\footnote{This is a footnote.}%
\footnote{This is a footnote.}%
\footnote{This is a footnote.}%
\footnote{This is a footnote.}%
\footnote{This is a footnote.}%
\footnote{This is a footnote.}%
\footnote{This is a footnote.}%
\footnote{This is a footnote.}%
\footnote{This is a footnote.}%
\footnote{This is a footnote.}%
\footnote{This is a footnote.}%
\footnote{This is a footnote.}%
\footnote{This is a footnote.}%
\footnote{This is a footnote.}%
\footnote{This is a footnote.}%
\footnote{This is a footnote.}%
\footnote{This is a footnote.}%
\footnote{This is a footnote.}%
\footnote{This is a footnote.}%
\footnote{This is a footnote.}%
\footnote{This is a footnote.}%
\footnote{This is a footnote.}%
\footnote{This is a footnote.}%
\footnote{This is a footnote.}%
\footnote{This is a footnote.}%
\footnote{This is a footnote.}%
\footnote{This is a footnote.}%
\footnote{This is a footnote.}%
\footnote{This is a footnote.}%
\footnote{This is a footnote.}%
\footnote{This is a footnote.}%
\footnote{This is a footnote.}%
\footnote{This is a footnote.}%
\footnote{This is a footnote.}%
\footnote{This is a footnote.}%
\footnote{This is a footnote.}%
\footnote{This is a footnote.}%
\footnote{This is a footnote.}%
\footnote{This is a footnote.}%
\footnote{This is a footnote.}%
\footnote{This is a footnote.}%
\footnote{This is a footnote.}%
\footnote{This is a footnote.}%
\footnote{This is a footnote.}%
\footnote{This is a footnote.}%
\footnote{This is a footnote.}%
\footnote{This is a footnote.}%
\footnote{This is a footnote.}%
\footnote{This is a footnote.}%

\chapter{Mathematics and Theorems}
\section{Math}
\[\pi=\sb{33}\]

\[
  \int\sin x\,\mathrm{d}x=\cos x + C
\]

\begin{proof}
Cubum autem in duos cubos, aut quadratoquadratum in duos quadratoquadratos
\& generaliter nullam in infinitum ultra quadratum potestatem in duos eiusdem
nominis fas est dividere cuius rei demonstrationem mirabilem sane detexi.
Hanc marginis exiguitas non caperet.
\end{proof}

\begin{definition}
Q.E.D. (also written QED) is an initialism of the Latin phrase quod erat
demonstrandum, meaning ``what was to be demonstrated'', or, less formally,
``thus it has been demonstrated''. The phrase is traditionally placed in
its abbreviated form at the end of a mathematical proof or philosophical
argument when the original proposition has been exactly restated as the
conclusion of the demonstration. The abbreviation thus signals the completion
of the proof.

The phrase quod erat demonstrandum is a translation into Latin from the
Greek ὅπερ ἔδει δεῖξαι (hoper edei deixai; abbreviated as ΟΕΔ). Translating
from the Latin into English yields, ``what was to be demonstrated''; however,
translating the Greek phrase ὅπερ ἔδει δεῖξαι produces a slightly different
meaning. Since the verb ``δείκνυμι'' also means to show or to prove, a
better translation from the Greek would read, ``The very thing it was required
to have shown.'' The phrase was used by many early Greek mathematicians,
including Euclid and Archimedes.
\end{definition}

\begin{lemma}
In the European Renaissance, scholars often wrote in Latin, and phrases
such as Q.E.D. were often used to conclude proofs.
\end{lemma}

\begin{proof}
\textbf{Ipso facto} is a Latin phrase, directly translated as ``by the fact itself'',
which means that a specific phenomenon is a \emph{direct} consequence, a
resultant \textit{effect}, of the action in question, instead of being brought
about by a previous action.
\begin{equation}
  \sum_{k=0}^{\infty} \frac{1}{x^k} = \int \sin x dx
\end{equation}
\end{proof}

\begin{proof}
\textbf{Ipso facto} is a Latin phrase, directly translated as ``by the fact itself'',
which means that a specific phenomenon is a \emph{direct} consequence, a
resultant \textit{effect}, of the action in question, instead of being brought
about by a previous action.
\begin{equation*}
  \sum_{k=0}^{\infty} \frac{1}{x^k} = \int \sin x dx
\end{equation*}
\end{proof}

\begin{lemma}
This is another beautiful lemma.
\end{lemma}

\chapter{Theorems (Continued)}
\newcounter{thm}

\fdunewtheorem[style=plain,qed=\ensuremath{\sin}]{p}{plain}
\fdunewtheorem[style=margin]{mm}{BREAKINGS}
\fdunewtheorem[style=change]{fduc}{Changing}
\fdunewtheorem[style=break]{fdub}{plain}
\fdunewtheorem[style=marginbreak]{mb}{BREAKINGS}
\fdunewtheorem[style=break]{cb}{Changing}
\fdunewtheorem*[style=plain]{np}{plain}
\fdunewtheorem*[style=margin]{nmm}{BREAKINGS}
\fdunewtheorem*[style=change]{nfduc}{Changing}
\fdunewtheorem*[style=break,qed=]{nfdub}{plain}
\fdunewtheorem*[style=marginbreak,qed={}]{nmb}{BREAKINGS}
\fdunewtheorem[style=break,counter=thm]{ncb}{Changing}

\newtheorem{prop}[thm]{Proposition}

\begin{prop}
  Right-angle triangle.
\end{prop}
\begin{prop}[Sphere]
  Right-angle triangle.
\end{prop}
\begin{prop}
  Right-angle triangle.
\end{prop}

\begin{p}
This is a THEOREM.
\[ \sum_{k=0}^{\infty} \frac{1}{x^k} = \int \sin x dx \]
\end{p}

\begin{mm}[Tomorrow morning]
This is a THEOREM.
\[ \sum_{k=0}^{\infty} \frac{1}{x^k} = \int \sin x dx \]
\end{mm}

\begin{fduc}
This is a THEOREM.
\[ \sum_{k=0}^{\infty} \frac{1}{x^k} = \int \sin x dx \]
\end{fduc}

\begin{fdub}
This is a THEOREM.
\[ \sum_{k=0}^{\infty} \frac{1}{x^k} = \int \sin x dx \]
\end{fdub}

\begin{mb}
This is a THEOREM.
\[ \sum_{k=0}^{\infty} \frac{1}{x^k} = \int \sin x dx \]
\end{mb}

\begin{cb}
This is a THEOREM.
\[ \sum_{k=0}^{\infty} \frac{1}{x^k} = \int \sin x dx \]
\end{cb}

%%%%%%%%%%%%%%%%%%%%%%%%%%%%%%%%%%%%%%%%%%%%%

\begin{np}
This is a THEOREM.
\[ \sum_{k=0}^{\infty} \frac{1}{x^k} = \int \sin x dx \]
\end{np}

\begin{nmm}[Another tomorrow morning]
This is a THEOREM.
\[ \sum_{k=0}^{\infty} \frac{1}{x^k} = \int \sin x dx \]
\end{nmm}

\begin{nfduc}
This is a THEOREM.
\[ \sum_{k=0}^{\infty} \frac{1}{x^k} = \int \sin x dx \]
\end{nfduc}

\begin{nfdub}
This is a THEOREM.
\[ \sum_{k=0}^{\infty} \frac{1}{x^k} = \int \sin x dx \]
\end{nfdub}

\begin{nmb}
This is a THEOREM.
\[ \sum_{k=0}^{\infty} \frac{1}{x^k} = \int \sin x dx \]
\end{nmb}

\begin{ncb}
This is a THEOREM.
\[ \sum_{k=0}^{\infty} \frac{1}{x^k} = \int \sin x dx \]
\end{ncb}

\begin{ncb}
This is a THEOREM.
\[ \sum_{k=0}^{\infty} \frac{1}{x^k} = \int \sin x dx \]
\end{ncb}

\begin{ncb}
This is a THEOREM.
\[ \sum_{k=0}^{\infty} \frac{1}{x^k} = \int \sin x dx \]
\end{ncb}

\begin{ncb}
This is a THEOREM.
\[ \sum_{k=0}^{\infty} \frac{1}{x^k} = \int \sin x dx \]
\end{ncb}

\chapter{Figures, Tables and Floats}
\section{title}
Myriad is a humanist sans-serif typeface designed by Robert Slimbach
and Carol Twombly for Adobe Systems. The typeface is best known for
its usage by Apple Inc., replacing Apple Garamond as Apple's corporate
font from 2002 to 2017. Myriad is easily distinguished from other
sans-serif fonts due to its special ``y'' descender (tail) and slanting
``e'' cut. Myriad is similar to Frutiger, although the italic is different;
Adrian Frutiger described the font as ``not badly done'' but felt that
the similarities had gone ``a little too far''. The later Segoe UI and
Corbel are also similar.

\begin{figure}[h]
  \centering
  \includegraphics[width=3cm]{../logo/fudan-emblem.pdf}
  \includegraphics[width=4cm]{../logo/fudan-emblem.pdf}
  \caption{This PostScript Type 1 font family was released after the
    original Myriad MM. It initially included four fonts in two weights,
    with complementary italics. All these Type 1 versions supported
    the ISO-Adobe character set; all were discontinued in the early 2000s.}
\end{figure}

It was a condensed version, released around 1998. The condensed fonts
comprise three weights, with complementary italics.

\section{title}
Myriad Web is a version of Myriad in TrueType font format, optimized
for onscreen use. It supports Adobe CE and Adobe Western 2 character
sets. Myriad Web comprises only five fonts: Myriad Web Pro Bold, Myriad
Web Pro Regular, Myriad Web Pro Condensed Italic, Myriad Web Pro Condensed,
Myriad Web Pro Italic. Myriad Web Pro is slightly wider than Myriad Pro,
while the width of Myriad Web Pro Condensed is between Myriad Pro
Condensed and Myriad Pro SemiCondensed.

The family is bundled as part of the Adobe Web Type Pro font pack.

\begin{table}[h]
  \centering
  \caption{A normal table}
  \begin{tabular}{ccc}
    \hline
    \bfseries Function & \bfseries Environment & \bfseries Code \\
    \hline
    Table     & tabular & \ttfamily \backslash begin\{tabular\} ... \backslash end\{tabular\} \\
    Figure    & figure  & \ttfamily \backslash begin\{figure\}  ... \backslash end\{figure\}  \\
    Centering & center  & \ttfamily \backslash begin\{center\}  ... \backslash end\{center\}  \\
    \hline
  \end{tabular}
\end{table}

Myriad Pro is the OpenType version of the original Myriad font family.
It first shipped in 2000, as Adobe moved towards the OpenType standard.
Additional designers were Christopher Slye and Fred Brady. Compared to
Myriad MM, it added support for Latin Extended, Greek, and Cyrillic
characters, as well as oldstyle figures.

Myriad Pro originally included thirty fonts in three widths and five
weights each, with complementary italics. A ``semi-condensed'' width
was added in early 2002, expanding the family to forty fonts in four
widths and five weights each, with complementary italics.

\section{title}
\kant[1-10]

\section{title}
\kant[21-30]

\section{title}
\kant[31-40]

\section{title}
\section{title}
\chapter{Chapter-HAHAHA}
\section{title}
\section{titvdale}
\section{title}
\section{titvsdle}
\section{title}
\section{ticadtle}
\section{title}

\end{document}
