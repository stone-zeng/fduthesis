\PassOptionsToPackage{log-declarations=false}{xparse}
\documentclass[twoside]{fduthesis-en}
\usepackage{kantlipsum}

\fdusetup{
  style = {
    % font = libertinus,
    % cjkfont = founder,
    % fullwidthstop,
    % footnotestyle = libertinus*,
    % automakecover = false,
    logo = {../logo/pdf/fudan-name-black.pdf},
    hyperlink = color,
    hyperlinkcolor = autumn,
    bibstyle = plainnat
  },
  info = {
    title = {自河南经乱关内阻饥兄弟离散各在一处},
    title* = {Measurements of the interaction between energetic
      photons and hadrons show that the interaction},
%    date = {2017年2月10日},
    author = {某某某},
    author* = {Xiangdong Zeng},
    supervisor = {陈丙丁 \quad 教授},
    instructors = {
      张五六 \quad 工程师,
      赵\quad 甲 \quad 工程师,
      王三四 \quad 讲\quad 师
    },
    major = {物理学},
    department = {凝聚态物理系},
    secretlevel = ii,
    secretyear = {三年},
    studentid = {14307110000},
    keywords = {\LaTeX, hih, vwvs, 物理, 中心法则, vsbd, 为侨服务},
    keywords* = {\LaTeX, hih, vwvs, Physics, Central Rule, vsbd},
    clc = {O414.1/65}
  }
}

\def\BibTeX{B\textsc{ib}\TeX}

\begin{document}

\frontmatter

\tableofcontents

\begin{abstract}
\LaTeX3 does not use @ as a ``letter'' for defining internal
macros. Instead, the symbols are used in internal macro names to
provide structure. The name of each function is divided into
logical units using separates the name of the function from the
argument specifier (``arg-spec''). This describes the arguments
expected by the function.
\end{abstract}

\begin{notation}
$\sin$      &  Sine \\
HPC         &  High Performance Computing \\
SMP         &  Symmetrical Multi-Processing \\
API         &  Application Programming Interface \\
PI          &  Polyimide \\
PBI         &  Polybenzimidazole \\
PY          &  Polypyrron \\
$\Delta G$  &  Activation Free Energy \\
$\chi$      &  Transmission Coefficient \\
$E$         &  Energy \\
$m$         &  Mass \\
$c$         &  Speed of Light \\
$P$         &  Possibility \\
$T$         &  Time \\
$v$         &  Velocity \\
Wikipedia   &  Wikipedia (/ˌwɪkᵻˈpiːdiə/ (About this sound listen)
               or /ˌwɪkiˈpiːdiə/ (About this sound listen)
               \emph{WIK-i-PEE-dee-ə}) is a free online encyclopedia
               with the aim to allow anyone to edit articles.
               Wikipedia is the largest and most popular general
               reference work on the Internet and is ranked among
               the ten most popular websites. Wikipedia is owned by
               the nonprofit Wikimedia Foundation.
\end{notation}

\mainmatter

\chapter{Text}

\section{Text and paragraph}

\textbf{Using \texttt{\string\cite}}
Helvetica or Neue Haas Grotesk is a widely used sans-serif typeface
developed in 1957 by Swiss typeface designer Max Miedinger with input
from Eduard Hoffmann \cite{test1}.
Helvetica is a neo-grotesque or realist design \cite{lzp1,crawfprd}, one influenced by
the famous 19th century typeface Akzidenz-Grotesk and other German
and Swiss designs \cite{hws,yufin,qgxxywxbzhjswyh}. Its use became a hallmark of the International
Typographic Style that emerged from the work of Swiss designers in
the 1950s and 60s, becoming one of the most popular typefaces of the
20th century. Over the years, a wide range of variants have been
released in different weights \cite{kanamori,wll}, widths and sizes, as well as matching
designs for a range of non-Latin alphabets. Notable features of
Helvetica as originally designed include a high x-height, the
termination of strokes on horizontal or vertical lines and an
unusually tight spacing between letters, which combine to give it a
dense, compact appearance \cite{test1,kanamori,wll,qgxxywxbzhjswyh}.

\textbf{Using \texttt{\string\citep}}
Developed by the Haas'sche Schriftgiesserei (Haas Type Foundry) of
Münchenstein, Switzerland, its release was planned to match a trend:
a resurgence of interest in turn-of-the-century grotesque typefaces
among European graphic designers that also saw the release of Univers
by Adrian Frutiger the same year \citep{caplan}. Hoffmann was the president of the
Haas Type Foundry, while Miedinger was a freelance graphic designer
who had formerly worked as a Haas salesman and designer \citep{frese,xadzkjdx,tachibana}.
Miedinger and Hoffmann set out to create a neutral typeface that had
great clarity, no intrinsic meaning in its form, and could be used
on a wide variety of signage. Originally named Neue Haas Grotesk
(New Haas Grotesque), it was rapidly licensed by Linotype and renamed
Helvetica, being similar to the Latin adjective for Switzerland,
Helvetia \citep{bawden,hws,yufin}. The font name was changed to Helvetica in 1960. A
feature-length film directed by Gary Hustwit was released in 2007
to coincide with the 50th anniversary of the typeface's introduction
in 1957 \citep{fg,ybj}.

\section{title}

\textbf{Using \texttt{\string\citet}}
Influences of Helvetica included Schelter-Grotesk and Haas'
Normal-Grotesk. Attracting considerable attention on its release
as Neue Haas Grotesk, Linotype adopted Neue Haas Grotesk for
widespread release \citet{buseck}.
In 1960, its name was changed by Haas' German parent company Stempel
to Helvetica (meaning Swiss in Latin) in order to make it more
marketable internationally \citet{udtfha,lml}. It comes from the Latin name for the
pre-Roman tribes of what became Switzerland. Intending to match the
success of Univers, Arthur Ritzel of Stempel redesigned Neue Haas
Grotesk into a larger family \citet{bcz,sftrj,babu,kanamori}. The design was popular, and rapidly
made available for phototypesetting systems as well as for the
original metal type. Many imitations and knock-offs were rapidly
created \citet{test1,kanamori,wll}.
In the late 1970s and 1980s, Linotype licensed its version to Xerox
and then Adobe and Apple, guaranteeing its importance in digital
printing by making it one of the core fonts of the PostScript page
description language. The rights to it are now held by Monotype
Imaging, which acquired Linotype; the advanced Neue Haas Grotesk
release (discussed below) was co-released with Font Bureau \citet{dcmes,wfz,wfz1,wfz2}.

\chapter{Dummy text}

\kant[1-15]

%\nocite{*}

\backmatter

\bibliography{test}

\end{document}
